% !TEX TS-program = XeLaTeX
% !TEX encoding = UTF-8 Unicode

%%%%%%%%%%%%%%%%%%%%%%%%%%%%%%%%%%%%%%%%%%%%%%%%%%%%%%%%%%%%%%%%%%%%%%
%
%  华东理工大学本硕博论文 XeLaTeX 模版 —— 主文件 main.tex
%
%  版本:1.0.0
%  最后更新:
%  修改者:Liu Qipei, Wu Xinying, liuqipei@hrbeu.edu.cn, y10170069@mail.ecust.edu.cn
%  修订者:
%  编译环境1:macOS 10.13  + TeXLive 2020
%  编译环境2:Windows 10  + TeXLive 2020
%
%%%%%%%%%%%%%%%%%%%%%%%%%%%%%%%%%%%%%%%%%%%%%%%%%%%%%%%%%%%%%%%%%%%%%%

\cnabstract{
本模板是在参考其他高校的硕博士论文模板基础上,
并按照哈尔滨工程大学学位论文格式规范开发的~\XeLaTeX~学位论文模板,
此目前已经基本满足了论文规范的要求,而且易用性良好。不过,可能还存在着
一些问题,欢迎大家积极反馈遇到的问题,以便不断对其进行改进。

当然这个模板仅仅是一个开始,希望有更多的人能够参与进来,
不断改进准确性、易用性和较好的可维护性。

本模板的目的旨在推广~\LaTeX~这一优秀的排版软件在论文撰写中的应用,
为广大同学提供一个方便、美观的论文模板,减少论文撰写格式方面的麻烦。

本文给出了利用本模板进行论文撰写的基本步骤,并介绍了一些常用~\XeLaTeX~
排版指令。
}

\cnkeywords{
学位论文;\XeLaTeX{}模版;使用说明
}

\enabstract{
This is a \LaTeX{} template of degree thesis of Harbin Engineering University,
which is built according to the required format.
}

\enkeywords{
Degree Paper; \XeLaTeX{} Template; manual
}