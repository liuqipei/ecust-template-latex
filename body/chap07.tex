% !TEX TS-program = XeLaTeX
% !TEX encoding = UTF-8 Unicode

%%%%%%%%%%%%%%%%%%%%%%%%%%%%%%%%%%%%%%%%%%%%%%%%%%%%%%%%%%%%%%%%%%%%%%
%
%  华东理工大学本硕博论文 XeLaTeX 模版 —— 主文件 main.tex
%
%  版本:1.0.0
%  最后更新:
%  修改者:Liu Qipei, Wu Xinying, liuqipei@hrbeu.edu.cn, y10170069@mail.ecust.edu.cn
%  修订者:
%  编译环境1:macOS 10.13  + TeXLive 2020
%  编译环境2:Windows 10  + TeXLive 2020
%
%%%%%%%%%%%%%%%%%%%%%%%%%%%%%%%%%%%%%%%%%%%%%%%%%%%%%%%%%%%%%%%%%%%%%%

\chapter{参考文献}
\label{chap07}

参考文献的引用一般有两种方式,即行间引用和上标引用。

行间引用使用\textbackslash lcite~\{~~\}~语句实现,其显示效果是这样的:例如文献\lcite{DXM2005}论述了什么什么,而文献\lcite{OJP1999,kelton2002,strawderman2001,LQL1999}则对这个那个进行了研究。

上标引用使用\textbackslash cite~\{~~\}~语句实现,下面这段文字是普通的上标引用格式

我们的一切知识都是从经验开始\cite{LQL1999},这是没有任何怀疑的\cite{DXM2005}\cite{DXM2000};
因为,如果不是对象激动我们的感官,一则由它们自己引起表象,一则使我们的知性活动运作起来,对这些表象加
以比较,把它们粘结或分开,\cite{OJP1999,OJP1991}这样把感性印象的原始素材加工成称之为经验的对象
知识,那么知识能力又该由什么来唤起活动呢?\cite{braun2007,kelton2002,strawderman2001,LQL1999}所以
按照时间,我们没有任何知识是先行于经验的,一切知识都是从经验开始的。

只要是中文文献,图书,期刊,会议,专利等等需要为每个条目增加一个域:
\begin{lstlisting}
  language={CN},
\end{lstlisting}

对于\cite{DXM2005}参考文献\cite{OJP1999},原先的bib文件是\scite{OJP1991}这样的:
\begin{lstlisting}
  @article{ LQL1999 ,
    title={ 康德何以步安瑟尔谟的后尘? },
    author={ 李秋零 },
    journal={ 中国人民大学学报 },
    volume={2},
    year={1999}
  }
\end{lstlisting}


但是由于是中文文献,需要增加一个语言域,就变成下列样式:
\begin{lstlisting}
  @article{ LQL1999,
    title={ 康德何以步安瑟尔谟的后尘? },
    author={ 李秋零 },
    language={CN},
    journal={ 中国人民大学学报 },
    volume={2},
    year={1999}
  }
\end{lstlisting}

\section{~BibTeX~文献文件的写法}

用在~\LaTeX~中的~\textsc{Bib}\kern-.08em\TeX~文献文件的扩展名为~bib,此模板中,该文件即为~reference.bib。bibtex.exe 命令根据~GBT7714-2005NLang-HIT.bst 文件定义的文献格式,将~reference.bib 中的文献数据转换为输出文档中的文献列表。GBT7714-2005NLang-HIT.bst 文件是在~\href{http://bbs.ctex.org/attachment.php?aid=MjA3MDh8ZDcyMjc2MTN8MTMyNTYzNjY4OHxhZTg4bkNCUVJiRzA0WmU3TmlMbVdTUVExa0xtV2puWWc0dkdqbVJhbTVMdy9mVQ\%3D\%3D}{GBT7714-2005NLang-UTF8.bst} 文件的基础上修改得到的,所做的唯一一处改动是将姓氏字母全部大写的英文作者名改为只首字母大写,以保证和\href{http://219.217.226.141/xuewei/guifan.doc}{《研究生学位论文撰写规范》}及其\href{http://219.217.226.141/xuewei/fanli.doc}{《研究生学位论文书写范例》}相一致。

bib 文件的编写方法可参考模板中已给出的例子,也可参考~\href{http://bbs.ctex.org/attachment.php?aid=MTk3OTd8NjY1ODc5OGV8MTMyNTY0MTEyMnxhZGZkYWpsa0I2RGZwNDR5Z1lyeStjb1dKRS8rTnJub3lvT2FkNDNJbHl1UWVkVQ\%3D\%3D}{GBT7714-2005.bst 说明文档20060919
} 中所给出的例子。

中文文献需要添加一个额外的~language 域,并使得域值非空,这样~bst 文件就能够判断此文献为中文文献,进而能正确地生成参考文献格式。

GBT7714-2005.bst 对于国标~GB/T 7714-2005 的文献分类如表~\ref{tab:entrytypes} 所示。对于每种文献类型的缺省类型,已经设置好相应的文献标识码,因此不需要输入相应的文献
标识码。扩展类型的文献则应再添加一个~TypeofLit 域,并需要将其域值改为相应的文献标识码。
\begin{table}[htbp]
\bicaption[tab:entrytypes]{}{GBT7714-2005.bst 的分类方式}{Table$\!$}{Classification method of GBT7714-2005.bst}
\vspace{0.5em}\centering\wuhao
\begin{tabular}{llll}
\toprule[1.5pt]
文献类型 & 缺省类型 & 扩展类型(需要手 & 主要特征\\
 &  & 工加入文献标识码) & \\
\midrule[1pt]
article & 文章[J] & 报纸中的析出文献[N] & 年,卷(期):页码\\
 &  & 在线文章[J/OL] & \\
book & 书[M] & 论文集、会议录[C] & \\
 &  & 在线书[M/OL] & \\
 &  & 汇编[G] & \\
inbook & 书的某几页[M] &  & \\
incollection & 书中析出的文章[M]// & 汇编的析出文献[G]// & 析出文献[文献标识码]//\\
 &  & 标准的析出文献[S]// & \\
proceedings &  &  & \\
inproceedings & 论文集、会议录中的 & 在线论文集、 & 析出文献[文献标识码]//\\
/conference & 析出文献[C]// & 会议录[C/OL]// & \\
mastersthesis & 毕业论文[D] &  & 类似book类\\
phdthesis & 毕业论文[D] &  & 类似book类\\
techreport & 科技报告[R] &  & 类似book类\\
misc &  & 杂项[],例如:专利[P] & 此类一般是网上文件,\\
 &  & 网上专利[P/OL] & 按照国标规定顺序\\
 &  & 网上电子公告[EB/OL] & 编码制时不输出年份\\
 &  & 磁盘[CP/DK] & \\
\bottomrule[1.5pt]
\end{tabular}
\end{table}

《研究生学位论文撰写规范》及《研究生学位论文书写范例》中所列英文参考文献例子中的文章名的每个实词首字母都大写,因此需要将英文参考文献的~title 域手动修改为每个实词首字母大写。

英文参考文献在~author 域中的作者名需要将姓置前,名置后。

\section{参考文献的引用}

需要将~main.tex 文件中的语句~\verb|\nocite{*}| 屏蔽掉,这样,文中未引用的参考文献就不会出现在文后的参考文献列表中。文中参考文献的引用方法:

\begin{itemize}
\item 行文引用请使用命令~\verb|\lcite{引用词}|,引用效果为“\lcite{OJP1999}”;
\item 上标引用请使用命令~\verb|\cite{引用词}|,引用效果为“\cite{OJP1999}”。
\end{itemize}
其中,上标引用命令~\verb|\lcite{}| 为本模板自定义的命令,其定义为
\begin{verbatim}
\DeclareRobustCommand\lcite{\@lcite}
\def\@lcite#1{\begingroup\let\@cite\NAT@citenum\citep{#1}\endgroup}
\end{verbatim}

\section*{本章小结}
参考文献排版方法介绍。
