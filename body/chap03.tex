% !TEX TS-program = XeLaTeX
% !TEX encoding = UTF-8 Unicode

%%%%%%%%%%%%%%%%%%%%%%%%%%%%%%%%%%%%%%%%%%%%%%%%%%%%%%%%%%%%%%%%%%%%%%
%
%  华东理工大学本硕博论文 XeLaTeX 模版 —— 主文件 main.tex
%
%  版本:1.0.0
%  最后更新:
%  修改者:Liu Qipei, Wu Xinying, liuqipei@hrbeu.edu.cn, y10170069@mail.ecust.edu.cn
%  修订者:
%  编译环境1:macOS 10.13  + TeXLive 2020
%  编译环境2:Windows 10  + TeXLive 2020
%
%%%%%%%%%%%%%%%%%%%%%%%%%%%%%%%%%%%%%%%%%%%%%%%%%%%%%%%%%%%%%%%%%%%%%%

\chapter{模版使用说明}
\label{chap03}

\section{个人信息}
使用模版的第一步当然是修改您的个人信息。与个人信息有关的内容位
于~{/preface/cover.tex}~文件中。对照着模版内容改就好了,没有什么难度。填
写专业、姓名和导师的时候注意添加适当空格,也就是$\sim$字符,以保持段落对齐。
这里默认论文提交日期为最后一次编译~main.tex~的日期,答辩日期需要手工设置。

\section{模版设置}

模板设置包括选择论文的学位类型、学科类型、汉字库和打印方式等,
这些内容的设置在~main.tex~文件中通过修改~\verb|\def|~命令实现。
\begin{itemize}
\item 学位论文类型选择 \\
学位论文类型可以是:~Doctor~(博士)、~Master~(硕士)和~Bachelor(学士)。
如论文选择“硕士”论文模板,则学位论文类型选择定义为:
\verb|\def\xuewei{Master}|

\item 定义学科 \\
本模板定义的学科包括:
\begin{table}[htbp]
  \bicaption[tab:xueke]{学科定义1}{学科定义}{Tab.}{Subject Definition}
  \centering
  \vspace{0.2cm}
  \zhongwu
  \begin{tabular}{cc}
    \toprule
    学科定义  & 学科类型  \\
    \midrule
    Engineering   & 工学 \\
    Science       & 科学 \\
    Management    & 管理 \\
    Arts          & 艺术 \\
    Philosophy    & 哲学 \\
    Economics     & 经济 \\
    Laws          & 法律 \\
    Education     & 教育 \\
    History       & 历史 \\
    \bottomrule
  \end{tabular}
\end{table}

如选择工学学科,则学科类型定义为:
\verb|\def\xueke{Engineering}|

\item 选择字体 \\
选择字体库,包括:\\
adobefont --Adobe 汉字库 \\
windowsfont --Windows 系统汉字库 \\
linuxfont --Linux 系统字库 \\

如选择~Adobe~汉字库,则选择类型选择定义为:
\verb|\def\fontselect{adobefont}|

\item 打印方式选择
论文打印方式包括~oneside~(单面打印)和~twoside~(双面打印)。
如选择双面打印方式,则打印方式选择定义为:
\verb|\def\oneortwoside{twoside}|
\end{itemize}

\section{中英文摘要、关键字}
中英文摘要和关键字也位于~{/preface/cover.tex}~文件中,分别定义
在cnabstract、 enabstract、cnkeywords和enkeywords中,替换成自己的即可。

这里附上研究生院对摘要和关键字的要求:
\begin{asparaenum}
\item “摘要”是摘要部分的标题,不可省略。论文摘要是学位论文的缩影,文字
  要简练、明确。内容要包括目的、方法、结果和结论。单位制一律换算成国际标
  准计量单位制,除特殊情况外,数字一律用阿拉伯数码。文中不允许出现插图,
  重要的表格可以写入;
\item 关键词请尽量用《汉语主题词表》等词表提供的规范词。关键词之间用全角
  分号间隔,末尾不加标点;
\item 英文摘要和中文摘要对应,但不要逐字翻译。英文关键字使用半角分号间隔,
  末尾同样不加标点。
\end{asparaenum}

\section{正文}

正文部分包括了绪论(chap01.tex)、正文内容章节
(chap02.tex、chap03.tex、chap04.tex、……)、结论(conclusion.tex)三个部分,
均位于body文件夹中。同时位于body文件夹下的还有Bib\TeX{}参考文献文件
(reference.bib)。

正文内容章节以chapXX.tex形式为文件名,从01开始计数,使得文件名序号即为章
节序号。这些正文内容章节需要依次写入main.tex文件中,格式
为~\verb|\include{body/chapXX}| 。

所有的图片放在figure文件夹中。

下面是研究生院对正文的要求:

“正文”不可省略。

正文是硕士学位论文的主体,要着重反映研究生自己的工作,要突出新的见解,例
如新思想、新观点、新规律、新研究方法、新结果等。正文一般可包括:理论分析;
试验装置和测试方法;对试验结果的分析讨论及理论计算结果的比较等。

正文要求论点正确,推理严谨,数据可靠,文字精练,条理分明,文字图表清晰整
齐,计算单位采用国务院颁布的《统一公制计量单位中文名称方案》中规定和名称。
各类单位、符号必须在论文中统一使用,外文字母必须注意大小写,正斜体。简化
字采用正式公布过的,不能自造和误写。利用别人研究成果必须附加说明。引用前
人材料必须引证原著文字。在论文的行文上,要注意语句通顺,达到科技论文所必
须具备的“正确、准确、明确”的要求。

\section{格式设置}
一般来说,采用本模板后不需要另外使用字体、字号、颜色等文字格式设置操作,
模板会根据内容自动选用合适的格式。但在某些情况下,如果需要特殊设置字体、
字号与颜色,那么可以使用下面这些方法进行设置。

\subsection{字体设置}
本模板预定义的汉字字体包括:{\song 宋体}、{\hei 黑体}、{\kai 楷体}和{\fs 仿宋},
每种字体还包括正体、斜体、粗体,而且可以实现复合效果,例如:\\
{\song 宋体  \textbf{加粗宋体} \textsl{斜体宋体} \textbf{\textsl{加粗斜体宋体}}} \\
{\hei 黑体   \textbf{加粗黑体} \textsl{斜体黑体} \textbf{\textsl{加粗斜体黑体}}} \\
{\kai 楷体   \textbf{加粗楷体} \textsl{斜体楷体} \textbf{\textsl{加粗斜体楷体}}} \\
{\fs  仿宋   \textbf{加粗仿宋} \textsl{斜体仿宋} \textbf{\textsl{加粗斜体仿宋}}} \\

设置字体的方法是在需要修改字体的文字前面加入字体定义指令,格式为\textbackslash~font,
其中\textbackslash~song表示宋体,\textbackslash~hei表示黑体,\textbackslash~kai表示楷体,
\textbackslash~fs表示仿宋,\textbackslash~xkai表示行楷。粗体的格式化指令为\textbackslash~textbf,
斜体的格式化指令为\textbackslash~textsl。
另外,可以用\{~~\}限定字体的设置范围,及将字体格式化指令和文字内容都放到\{~~\}内,
这样括号外面的内容格式自动恢复为以前的格式。

上面字体显示效果的实现代码为:

\begin{lstlisting}
{\song 宋体  \textbf{加粗宋体} \textsl{斜体宋体}
 \textbf{\textsl{加粗斜体宋体}}} \\
{\hei 黑体   \textbf{加粗黑体} \textsl{斜体黑体}
 \textbf{\textsl{加粗斜体黑体}}} \\
{\kai 楷体   \textbf{加粗楷体} \textsl{斜体楷体}
 \textbf{\textsl{加粗斜体楷体}}} \\
{\fs  仿宋   \textbf{加粗仿宋} \textsl{斜体仿宋}
 \textbf{\textsl{加粗斜体仿宋}}} \\
\end{lstlisting}

\subsection{字号设置}
设置字号的方法为:
\begin{lstlisting}
\xiaowu   \textbackslash xiaowu~  小五,默认单倍行距 \\
\wuhao    \textbackslash wuhao~   五号,默认单倍行距 \\
\zhongwu  \textbackslash zhongwu~ 五号,默认1.5 倍行距 \\
\dawu     \textbackslash dawu~    五号,默认1.75倍行距 \\
\xiaosi   \textbackslash xiaosi~  小四,默认1.25倍行距 \\
\daxiaosi \textbackslash daxiaosi~小四,默认1.5 倍行距 \\
\sihao    \textbackslash sihao~   四号,默认1.25倍行距 \\
\xiaosan  \textbackslash xiaosan~ 小三,默认1.25 倍行距 \\
\sanhao   \textbackslash sanhao~  三号,默认1.25 倍行距 \\
\xiaoer   \textbackslash xiaoer~  小二,默认1.25 倍行距 \\
\erhao    \textbackslash erhao~   二号,默认1.25 倍行距 \\
\xiaoyi   \textbackslash xiaoyi~  小一,默认1.25 倍行距 \\
\yihao    \textbackslash yihao~   一号,默认1.5  倍行距 \\
\end{lstlisting}

打印效果如下:

 \begin{flushleft}
 {
\xiaowu   \textbackslash xiaowu~  小五,默认单倍行距 \\
\wuhao    \textbackslash wuhao~   五号,默认单倍行距 \\
\zhongwu  \textbackslash zhongwu~ 五号,默认1.5 倍行距 \\
\dawu     \textbackslash dawu~    五号,默认1.75倍行距 \\
\xiaosi   \textbackslash xiaosi~  小四,默认1.25倍行距 \\
\daxiaosi \textbackslash daxiaosi~小四,默认1.5 倍行距 \\
\sihao    \textbackslash sihao~   四号,默认1.25倍行距 \\
\xiaosan  \textbackslash xiaosan~ 小三,默认1.25 倍行距 \\
\sanhao   \textbackslash sanhao~  三号,默认1.25 倍行距 \\
\xiaoer   \textbackslash xiaoer~  小二,默认1.25 倍行距 \\
\erhao    \textbackslash erhao~   二号,默认1.25 倍行距 \\
\xiaoyi   \textbackslash xiaoyi~  小一,默认1.25 倍行距 \\
\yihao    \textbackslash yihao~   一号,默认1.5  倍行距 \\
}
\end{flushleft}

\subsection{颜色设置}
设置文字颜色的方法为:
\begin{lstlisting}
\definecolor {myrgb}{rgb}{0.25, 0.5, 0.25}
\definecolor {mycmyk}{cmyk}{1, 0.8, 0.2, 0.1}

\hei \textcolor{black}  {这是预定义颜色-黑色 balck}  \\
\hei \textcolor{red}    {这是预定义颜色-红色 red}  \\
\hei \textcolor{blue}   {这是预定义颜色-蓝色 blue}  \\
\hei \textcolor{yellow} {这是预定义颜色-黄色 yellow}  \\
\hei \textcolor{myrgb}  {这是自定义RGB颜色 myrgb}  \\
\hei \textcolor{mycmyk} {这是自定义CMYK颜色 mycmyk}  \\
\end{lstlisting}


文字颜色设置打印效果:
\begin{flushleft}
\xiaosan
{
    \definecolor {myrgb}{rgb}{0.25, 0.5, 0.25}
    \definecolor {mycmyk}{cmyk}{1, 0.8, 0.2, 0.1}

    \hei \textcolor{black}  {这是预定义颜色-黑色 balck}  \\
    \hei \textcolor{red}    {这是预定义颜色-红色 red}  \\
    \hei \textcolor{blue}   {这是预定义颜色-蓝色 blue}  \\
    \hei \textcolor{yellow} {这是预定义颜色-黄色 yellow}  \\
    \hei \textcolor{myrgb}  {这是自定义RGB颜色 myrgb}  \\
    \hei \textcolor{mycmyk} {这是自定义CMYK颜色 mycmyk}  \\
}
\end{flushleft}

\section*{本章小结}
简单介绍模板使用方法和文字格式化方法。
