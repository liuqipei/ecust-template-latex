% !TEX TS-program = XeLaTeX
% !TEX encoding = UTF-8 Unicode

%%%%%%%%%%%%%%%%%%%%%%%%%%%%%%%%%%%%%%%%%%%%%%%%%%%%%%%%%%%%%%%%%%%%%%
%
%  华东理工大学本硕博论文 XeLaTeX 模版 —— 主文件 main.tex
%
%  版本:1.0.0
%  最后更新:
%  修改者:Liu Qipei, Wu Xinying, liuqipei@hrbeu.edu.cn, y10170069@mail.ecust.edu.cn
%  修订者:
%  编译环境1:macOS 10.13  + TeXLive 2020
%  编译环境2:Windows 10  + TeXLive 2020
%
%%%%%%%%%%%%%%%%%%%%%%%%%%%%%%%%%%%%%%%%%%%%%%%%%%%%%%%%%%%%%%%%%%%%%%

% 页面设置
\usepackage{geometry}
\usepackage{indentfirst}                         % 首行缩进宏包
\usepackage[center]{titlesec}                    % 控制标题的宏包
\usepackage{titletoc}                            % 控制目录的宏包
\usepackage{fancyhdr}                            % 自定义页眉页脚
\usepackage{fancyref}                            % 引用链接属性
\usepackage[perpage,symbol]{footmisc}            % 脚注控制
\usepackage{layouts}                             % 打印当前页面格式的宏包
\usepackage{paralist}                            % 一种换行不缩进的列表格式,asparaenum,inparaenum 等
\usepackage[shortlabels]{enumitem}               % 列表格式
\usepackage{fancyvrb}                            % 原样输出
\usepackage[amsmath,thmmarks,hyperref]{ntheorem} % 定理类环境宏包
\usepackage{type1cm}                             % 控制字体的大小

% 表格处理
\usepackage{booktabs}   % 三线表
\usepackage{multirow}   % 表格多行处理
\usepackage{diagbox}    % 斜线表头
\usepackage{tabularx}   % 表格折行
\usepackage{siunitx}    % 国际单位,小数点对齐


% 图形相关
\usepackage{graphicx}          % 请在引用图片时务必给出后缀名
\usepackage[x11names]{xcolor}  % 支持彩色
\usepackage[below]{placeins}   % 浮动图形控制宏包
\usepackage{rotating}          % 图形和表格的控制
\usepackage{picinpar}
\usepackage{setspace}          % 定制表格和图形的多行标题行距
\usepackage{subfigure}           % 插入子图形
\usepackage[subfigure]{ccaption} % 插图表格的双语标题
\usepackage{tikz}
\usepackage{pifont}           % 带圈数字①-⑩

% 其他
\usepackage{calc}   % 在 tex 文件中具有一些计算功能,主要用在页面控制。

%\usepackage[numbers,sort&compress,square,super]{natbib} %参考文献
\usepackage[numbers,sort&compress,square]{natbib} %参考文献

\usepackage{hypernat}
\usepackage{bibentry}

\usepackage{listings}         % 源代码展示
\lstset{%
  language=TeX,
  defaultdialect=empty,
  basicstyle=\ttfamily\small,
  backgroundcolor=\color{LightSteelBlue1},
  keywordstyle=\color{blue},
  showspaces=false,
  showstringspaces=false,
  showtabs=false,
  tabsize=2,breakatwhitespace=false,
  columns=flexible}

% 生成有书签的pdf及其开关, 该宏包应放在所有宏包的最后, 宏包之间有冲突
\usepackage[xetex,
            bookmarksnumbered=true,
            bookmarksopen=true,
            colorlinks=true,
            %pdfborder={0 0 1},
            citecolor=blue, % 文献标柱颜色
            linkcolor=black, % 锚点颜色
            anchorcolor=green, % 超链颜色
            urlcolor=blue,
            breaklinks=true,
			CJKbookmarks=true,
            naturalnames  %与algorithm2e宏包协调
            ]{hyperref}
