% !TEX TS-program = XeLaTeX
% !TEX encoding = UTF-8 Unicode

%%%%%%%%%%%%%%%%%%%%%%%%%%%%%%%%%%%%%%%%%%%%%%%%%%%%%%%%%%%%%%%%%%%%%%
%
%  华东理工大学本硕博论文 XeLaTeX 模版 —— 主文件 main.tex
%
%  版本:1.0.0
%  最后更新:
%  修改者:Liu Qipei, Wu Xinying, liuqipei@hrbeu.edu.cn, y10170069@mail.ecust.edu.cn
%  修订者:
%  编译环境1:macOS 10.13  + TeXLive 2020
%  编译环境2:Windows 10  + TeXLive 2020
%
%%%%%%%%%%%%%%%%%%%%%%%%%%%%%%%%%%%%%%%%%%%%%%%%%%%%%%%%%%%%%%%%%%%%%%

%%%%%%%%%%%%%%%%%%%%%%%%%%%%%%%%%%%%%%%%%%%%%%%%%%%%%%%%%%%%%%%%%%%%%%
% 页面设置
%%%%%%%%%%%%%%%%%%%%%%%%%%%%%%%%%%%%%%%%%%%%%%%%%%%%%%%%%%%%%%%%%%%%%%
% A4 纸张
\setlength{\paperwidth}{210mm}
\setlength{\paperheight}{297mm}

% 设置正文尺寸大小
\setlength{\textwidth}{160mm}
\setlength{\textheight}{245mm}

% 设置正文区在正中间
\newlength \mymargin
\setlength{\mymargin}{(\paperwidth-\textwidth)/2}
\setlength{\oddsidemargin}{(\mymargin)-1in}
\setlength{\evensidemargin}{(\mymargin)-1in}

% 设置正文区偏移量,奇数页向右偏,偶数页向左偏
\newlength \myshift
\setlength{\myshift}{0mm}    % 双面打印的奇偶页偏移值,可根据需要修改,建议小于 5mm
\addtolength{\oddsidemargin}{\myshift}
\addtolength{\evensidemargin}{-\myshift}

% 页眉页脚相关距离设置
\setlength{\voffset}{-5.4mm} % 设置水平基线位置
\setlength{\topmargin}{0mm}  % 设置页眉距水平基线位置
\setlength{\headheight}{5mm} % 设置页眉高度
\setlength{\headsep}{8mm}    % 页眉与正文的距离
\setlength{\footskip}{6mm}   % 页脚与正文距离

% 公式的精调
\allowdisplaybreaks[4]  % 可以让公式在排不下的时候分页排,这可避免页面有大段空白。

% 下面这组命令使浮动对象的缺省值稍微宽松一点,从而防止幅度
% 对象占据过多的文本页面,也可以防止在很大空白的浮动页上放置很小的图形。
\renewcommand{\topfraction}{0.9999999}
\renewcommand{\textfraction}{0.0000001}
\renewcommand{\floatpagefraction}{0.9999}

% defaultfont 默认字体命令
\def\defaultfont{\renewcommand{\baselinestretch}{1.25} \daxiaosi}
%  \fontsize{12pt}{15pt}\selectfont}

% 设置目录字体和行间距
\def\defaultmenufont{\renewcommand{\baselinestretch}{1.22} \xiaosi}
%  \fontsize{12pt}{15pt}\selectfont}

% 固定距离内容填入及下划线
\makeatletter
\newcommand\fixeddistanceleft[2][10mm]{{\hb@xt@ #1{#2\hss}}}
\newcommand\fixeddistancecenter[2][10mm]{{\hb@xt@ #1{\hss#2\hss}}}
\newcommand\fixeddistanceright[2][10mm]{{\hb@xt@ #1{\hss#2}}}
\newcommand\fixedunderlineleft[2][10mm]{\underline{\hb@xt@ #1{#2\hss}}}
\newcommand\fixedunderlinecenter[2][10mm]{\underline{\hb@xt@ #1{\hss#2\hss}}}
\newcommand\fixedunderlineright[2][10mm]{\underline{\hb@xt@ #1{\hss#2}}}
\makeatother

%%%%%%%%%%%%%%%%%%%%%%%%%%%%%%%%%%%%%%%%%%%%%%%%%%%%%%%%%%%%%%%%%%%%%%
% 标题环境相关
%%%%%%%%%%%%%%%%%%%%%%%%%%%%%%%%%%%%%%%%%%%%%%%%%%%%%%%%%%%%%%%%%%%%%%
%判断单双面打印类型
\newif\ifoneortwosidetwoside
\newif\ifoneortwosideoneside

\def\temp{twoside}
\ifx\temp\oneortwoside
  \oneortwosidetwosidetrue  \oneortwosideonesidefalse
\fi

\def\temp{oneside}
\ifx\temp\oneortwoside
  \oneortwosidetwosidefalse  \oneortwosideonesidetrue
\fi

%判断论文类型
% 声明三个论文类型逻辑型变量
\newif\ifxueweidoctor
\newif\ifxueweimaster
\newif\ifxueweibachelor

% 根据 \xuewei 的定义为 \xueweidoctor \xueweimaster \xueweibachelor 赋值
\def\temp{Doctor}
\ifx\temp\xuewei % \ifx 用于判断两个变量是否匹配
  \xueweidoctortrue  \xueweimasterfalse \xueweibachelorfalse
\fi
\def\temp{Master}
\ifx\temp\xuewei
  \xueweidoctorfalse  \xueweimastertrue \xueweibachelorfalse
\fi
\def\temp{Bachelor}
\ifx\temp\xuewei
  \xueweidoctorfalse  \xueweimasterfalse \xueweibachelortrue
\fi

\ifxueweidoctor
  \newcommand{\cnxuewei}{博士}
  \newcommand{\enxuewei}{Doctor}
\fi

\ifxueweimaster
  \newcommand{\cnxuewei}{硕士}
  \newcommand{\enxuewei}{Master}
\fi

\ifxueweibachelor
  \newcommand{\cnxuewei}{学士}
  \newcommand{\enxuewei}{Bachelor}
\fi

%定义 学科 学位
\def \xuekeEngineering {Engineering}
\def \xuekeScience     {Science}
\def \xuekeManagement  {Management}
\def \xuekeArts        {Arts}
\def \xuekePhilosophy  {Philosophy}
\def \xuekeEconomics   {Economics}
\def \xuekeLaws        {Laws}
\def \xuekeEducation   {Education}
\def \xuekeHistory     {History}


\ifx \xueke \xuekeEngineering
\newcommand{\cnxueke}{工学}
\newcommand{\enxueke}{Engineering}
\newcommand{\enxk}   {Eng}
\fi

\ifx \xueke \xuekeScience
\newcommand{\cnxueke}{理学}
\newcommand{\enxueke}{Science}
\newcommand{\enxk}   {Sci}
\fi

\ifx \xueke \xuekeManagement
\newcommand{\cnxueke}{管理学}
\newcommand{\enxueke}{Management}
\newcommand{\enxk}   {Man}
\fi

\ifx \xueke \xuekeArts
\newcommand{\cnxueke}{文学}
\newcommand{\enxueke}{Arts}
\newcommand{\enxk}   {Art}
\fi

\ifx \xueke \xuekePhilosophy
\newcommand{\cnxueke}{哲学}
\newcommand{\enxueke}{Philosophy}
\newcommand{\enxk}   {Phi}
\fi

\ifx \xueke \xuekeEconomics
\newcommand{\cnxueke}{经济学}
\newcommand{\enxueke}{Economics}
\newcommand{\enxk}   {Eco}
\fi

\ifx \xueke \xuekeLaws
\newcommand{\cnxueke}{法学}
\newcommand{\enxueke}{Laws}
\newcommand{\enxk}   {Law}
\fi

\ifx \xueke \xuekeEducation
\newcommand{\cnxueke}{教育学}
\newcommand{\enxueke}{Education}
\newcommand{\enxk}   {Edu}
\fi

\ifx \xueke \xuekeHistory
\newcommand{\cnxueke}{历史学}
\newcommand{\enxueke}{History}
\newcommand{\enxk}   {His}
\fi

% 定义、定理等环境
\theoremstyle{plain}
\theoremheaderfont{\hei\bf}
\theorembodyfont{\song\rmfamily}
\newtheorem{definition}{\hei 定义}[chapter]
\newtheorem{example}{\hei 例}[chapter]
\newtheorem{algorithm}{\hei 算法}[chapter]
\newtheorem{theorem}{\hei 定理}[chapter]
\newtheorem{axiom}{\hei 公理}[chapter]
\newtheorem{proposition}[theorem]{\hei 命题}
\newtheorem{property}{\hei 性质}
\newtheorem{lemma}[theorem]{\hei 引理}
\newtheorem{corollary}{\hei 推论}[chapter]
\newtheorem{remark}{\hei 注解}[chapter]
\newenvironment{proof}{\hei{证明} }{\hfill $\square$ \vskip 4mm}

% 目录标题
\renewcommand{\contentsname}{\hfill \hei 目~~~~录 \hfill}
\renewcommand{\listfigurename}{\hfill 插~图~目~录 \hfill}
\renewcommand{\listtablename}{\hfill 表~格~目~录 \hfill}
\renewcommand{\bibname}{\hfill 参~考~文~献 \hfill}
\renewcommand\appendixname{附~录}

%%%%%%%%%%%%%%%%%%%%%%%%%%%%%%%%%%%%%%%%%%%%%%%%%%%%%%%%%%%%%%%%%%%%%%
% 段落章节相关
%%%%%%%%%%%%%%%%%%%%%%%%%%%%%%%%%%%%%%%%%%%%%%%%%%%%%%%%%%%%%%%%%%%%%%
\setcounter{secnumdepth}{4}
\setcounter{tocdepth}{4}
\setcounter{chapter}{0}

% 设置章、节、小节、小小节的间距
%\titleformat{\chapter}[hang]{\normalfont\xiaosan\hei\sf}{\xiaosan\thechapter}{1em}{\xiaosan}
\titleformat{\chapter}{\centering\sihao\hei\sf}{第\,\thechapter\,章}{1em}{} % 设置中文章格式中央对齐:第  章
\titlespacing{\chapter}{0pt}{-3ex  plus .1ex minus .2ex}{3.3ex}
\titleformat{\section}[hang]{\xiaosi\song\sf}{\xiaosi\thesection}{1em}{}{}
\titlespacing{\section}{0pt}{0.5em}{0.5em}
\titleformat{\subsection}[hang]{\xiaosi\song}{\xiaosi\thesubsection}{1em}{}{}
\titlespacing{\subsection}{0pt}{0.5em}{0.3em}
\titleformat{\subsubsection}[hang]{\xiaosi\hei\sf}{\xiaosi\thesubsubsection}{1em}{}{}
\titlespacing{\subsubsection}{0pt}{0.3em}{0pt}

% 缩小目录中各级标题之间的缩进
% \dottedcontents{<section>}[<left>]{<above>}{<labelwidth>}{<leaderwidth>}
\dottedcontents{chapter}[3mm]{\vspace{0.2em}}{1.0em}{5pt}
\dottedcontents{section}[8mm]{}{1.8em}{5pt}
\dottedcontents{subsection}[23mm]{}{2.7em}{5pt}
\dottedcontents{subsubsection}[33mm]{}{3.4em}{5pt}

% 设置目录中各级标题之间的缩进
\makeatletter
\renewcommand*{\l@chapter}{\@dottedtocline{0}{0em}{1em}}% 细点\@dottedtocline  粗点\@dottedtoclinebold
\renewcommand*{\l@section}{\@dottedtocline{1}{0em}{2em}}
\renewcommand*{\l@subsection}{\@dottedtocline{2}{0em}{3em}}
\renewcommand*{\l@subsubsection}{\@dottedtocline{3}{0em}{4em}}

% 设置章标题格式
% \titlecontents{章节名称}[左端距离]{标题字体、与上文间距等}{标题序号}{空}{引导符和页码}[与下文间距]
% 设置目录的点间距
\titlecontents{chapter}[3.8em]{\hspace{-3.8em}\hei}{第~\thecontentslabel~章~~\hspace{.6em}}{}{\titlerule*[8pt]{.}\contentspage}

% 段落之间的竖直距离
\setlength{\parskip}{1.2pt}
% 段落缩进
\setlength{\parindent}{24pt}
% 定义行距
\renewcommand{\baselinestretch}{1.25}
% 参考文献条目间行间距
\setlength{\bibsep}{2pt}


%%%%%%%%%%%%%%%%%%%%%%%%%%%%%%%%%%%%%%%%%%%%%%%%%%%%%%%%%%%%%%%%%%%%%%
% 页眉页脚设置
%%%%%%%%%%%%%%%%%%%%%%%%%%%%%%%%%%%%%%%%%%%%%%%%%%%%%%%%%%%%%%%%%%%%%%

\newcommand{\makeheadrule}{%
    \rule[12pt]{\textwidth}{0.5pt} \\[-23pt]
%    \rule{\textwidth}{2.0pt}
  \vskip-.8\baselineskip}

\makeatletter
\renewcommand{\headrule}{%
  {\if@fancyplain\let\headrulewidth\plainheadrulewidth\fi
    \makeheadrule}}

\pagestyle{fancyplain}

%去掉章节标题中的数字
%%不要注销这一行,否则页眉会变成:“第1章1  绪论”样式
\renewcommand{\chaptermark}[1]{\markboth{第\thechapter 章~~~~\ #1}{}}
\fancyhf{}

% 附录设置:附录不编章节号,但列入目录和页眉
\renewcommand{\appendix}[1]{%
    \chapter*{#1}%
    \addcontentsline{toc}{chapter}{#1}%
    \markboth{#1}{#1}
}

%在book文件类别下,\leftmark自动存录各章之章名,\rightmark记录节标题
%根据单双面打印设置不同的页眉;
\fancyhead[CO]{\song\xiaosi{\sihao\kai\textbf{\@cnuniversty}}~~\cnxuewei 学位论文\hfill{第$~\thepage~$页}}
\fancyhead[CE]{\song\xiaosi{第$~\thepage~$页\hfill{\sihao\kai\textbf{\@cnuniversty}}~~\cnxuewei 学位论文}}
%\fancyfoot[C,C]{\xiaosi$-$~\thepage~$-$}

%偶数页为空白页面时处理
%\makeatletter
%\def\cleardoublepage{\clearpage\if@twoside \ifodd\c@page\else%
%  \hbox{}%
%%  \thispagestyle{empty}%   % 清除页眉、页脚
%  \vspace*{80mm}
%  \centerline{\xiaoer\song {{(此页无正文)}}}
%  \centerline{\xiaosi\song {{(THIS PAGE IS INTENTIONALLY LEFT BLANK)}}}
%  \newpage%
%  \if@twocolumn\hbox{}\newpage\fi\fi\fi}

\makeatletter
\def\cleardoublepage{\newpage}

%%%%%%%%%%%%%%%%%%%%%%%%%%%%%%%%%%%%%%%%%%%%%%%%%%%%%%%%%%%%%%%%%%%%%%
% 列表环境设置
%%%%%%%%%%%%%%%%%%%%%%%%%%%%%%%%%%%%%%%%%%%%%%%%%%%%%%%%%%%%%%%%%%%%%%

\setlist[enumerate]{(1),itemsep=-5pt,topsep=0mm,labelindent=\parindent,leftmargin=*}

%%%%%%%%%%%%%%%%%%%%%%%%%%%%%%%%%%%%%%%%%%%%%%%%%%%%%%%%%%%%%%%%%%%%%%
% 国际单位,以点连接。
%%%%%%%%%%%%%%%%%%%%%%%%%%%%%%%%%%%%%%%%%%%%%%%%%%%%%%%%%%%%%%%%%%%%%%
\sisetup{inter-unit-product = { }\cdot{ }}

%%%%%%%%%%%%%%%%%%%%%%%%%%%%%%%%%%%%%%%%%%%%%%%%%%%%%%%%%%%%%%%%%%%%%%
% 参考文献的处理
%%%%%%%%%%%%%%%%%%%%%%%%%%%%%%%%%%%%%%%%%%%%%%%%%%%%%%%%%%%%%%%%%%%%%%

% \addtolength{\bibsep}{-0.5 em}      % 缩小参考文献间的垂直间距
% 上标引用,比\cite位置更偏上、字号稍小
\DeclareRobustCommand\scite{\@scite}
\def\@scite#1{\textsuperscript{\cite{#1}}}
% 行间引用,与正文格式一致
\DeclareRobustCommand\lcite{\@lcite}
\def\@lcite#1{\begingroup\let\@cite\NAT@citenum\citep{#1}\endgroup}

\setlength{\bibhang}{2em}
\bibpunct{[}{]}{,}{s}{}{}


%%%%%%%%%%%%%%%%%%%%%%%%%%%%%%%%%%%%%%%%%%%%%%%%%%%%%%%%%%%%%%%%%%%%%%
%   其他设置
%%%%%%%%%%%%%%%%%%%%%%%%%%%%%%%%%%%%%%%%%%%%%%%%%%%%%%%%%%%%%%%%%%%%%%
% 使图编号为 7-1 的格式 %\protect{~}
\renewcommand{\thefigure}{\arabic{chapter}.\arabic{figure}}
% 使子图编号为 (a)的格式
\renewcommand{\thesubfigure}{(\alph{subfigure})}
% 使子图引用为 7-1 (a) 的格式,母图编号和子图编号之间用~加一个空格
\renewcommand{\p@subfigure}{\thefigure~}
% 使表编号为 7-1 的格式
\renewcommand{\thetable}{\arabic{chapter}.\arabic{table}}
% 使公式编号为 7-1 的格式
\renewcommand{\theequation}{\arabic{chapter}-\arabic{equation}}

%插图索引格式: 图 x. 图标题 ......页码
\renewcommand\listoffigures{%
    \chapter*{\listfigurename}%
    \addcontentsline{toc}{chapter}{插图目录}
    \markboth{\listfigurename}{\listfigurename}
    \renewcommand{\numberline}[1]{图~##1~~}
    \@starttoc{lof}%
    }

%表格索引格式: 表 x. 表标题 ......页码
\renewcommand\listoftables{%
    \chapter*{\listtablename}%
    \addcontentsline{toc}{chapter}{表格目录}
    \markboth{\listtablename}{\listtablename}
    \renewcommand{\numberline}[1]{表~##1~~}
    \@starttoc{lot}%
    }

%%%%%%%%%%%%%%%%%%%%%%%%%%%%%%%%%%%%%%%%%%%%%%%%%%%%%%%%%%%%%%%%%%%%%%
%   图形表格
%%%%%%%%%%%%%%%%%%%%%%%%%%%%%%%%%%%%%%%%%%%%%%%%%%%%%%%%%%%%%%%%%%%%%%
\renewcommand{\figurename}{\song\textbf{图}}
\renewcommand{\tablename}{\song\textbf{表}}
% \captionstyle{\centering}
% \hangcaption
\captiondelim{\hspace{1em}}
\captiondelim{\hspace{1em}}
\captionnamefont{\zhongwu\song}
\captiontitlefont{\zhongwu\song}
\setlength{\abovecaptionskip}{0pt}
\setlength{\belowcaptionskip}{0pt}
%\captionsetup[bi-first]{font=sf}
\newcommand{\mycaption}[3]{\bicaption[#1]{#2}{\textbf{#2}}{Table}{#3}}

\newcommand{\tablepage}[2]{\begin{minipage}{#1}\vspace{0.5ex} #2 \vspace{0.5ex}\end{minipage}}
\newcommand{\returnpage}[2]{\begin{minipage}{#1}\vspace{0.5ex} #2 \vspace{-1.5ex}\end{minipage}}

%%%%%%%%%%%%%%%%%%%%%%%%%%%%%%%%%%%%%%%%%%%%%%%%%%%%%%%%%%%%%%%%%%%%%%
%   脚注格式设置
%%%%%%%%%%%%%%%%%%%%%%%%%%%%%%%%%%%%%%%%%%%%%%%%%%%%%%%%%%%%%%%%%%%%%%
  %使用pifont包里面ding产生带圈的数字1~10
\newcommand\chnnocirc[1]{%
\ifcase#1 a \or {\ding{172}} \or {\ding{173}} \or {\ding{174}} \or {\ding{175}} \or {\ding{176}} \or {\ding{177}} \or {\ding{178}} \or {\ding{179}} \or {\ding{180}}\fi}
\renewcommand{\thefootnote}{\chnnocirc{\arabic{footnote}}}

% 自定义一个空命令,用于注释掉文本中不需要的部分。
\newcommand{\comment}[1]{}

% 双语章节重新定义BiChapter、BiSection等命令,可实现标题手动换行,但不影响目录
\def\BiChapter{\relax\@ifnextchar [{\@BiChapter}{\@@BiChapter}}
\def\@BiChapter[#1]#2#3{\chapter[#1]{#2}
    \addcontentsline{toe}{chapter}{\bfseries \xiaosi Chapter \thechapter\hspace{0.5em} #3}}
\def\@@BiChapter#1#2{\chapter{#1}
    \addcontentsline{toe}{chapter}{\bfseries \xiaosi Chapter \thechapter\hspace{0.5em}{\boldmath #2}}}

\newcommand{\BiSection}[2]
{   \section{#1}
    \addcontentsline{toe}{section}{\protect\numberline{\csname thesection\endcsname}#2}
}

\newcommand{\BiSubsection}[2]
{    \subsection{#1}
    \addcontentsline{toe}{subsection}{\protect\numberline{\csname thesubsection\endcsname}#2}
}

\newcommand{\BiSubsubsection}[2]
{    \subsubsection{#1}
    \addcontentsline{toe}{subsubsection}{\protect\numberline{\csname thesubsubsection\endcsname}#2}
}

\newcommand{\BiAppendix}[2] % 该附录命令适用于发表文章,简历等
{\phantomsection
\markboth{#1}{#1}
\addcontentsline{toc}{chapter}{\xiaosi #1}
\addcontentsline{toe}{chapter}{\bfseries \xiaosi #2}  \chapter*{#1}
}

\newcommand{\BiAppChapter}[2]    % 该附录命令适用于有章节的完整附录
{\phantomsection
 \chapter{#1}
 \addcontentsline{toe}{chapter}{\bfseries \xiaosi Appendix \thechapter~~#2}
}

\def\engcontentsname{\uppercase{CONTENTS}}
\newcommand\tableofengcontents{
   \pdfbookmark[0]{\uppercase{CONTENTS}}{encontent}
   \chapter*{\engcontentsname  %chapter*上移一行,避免在toc中出现
       \@mkboth{%
          \engcontentsname}{\engcontentsname}}
   \@starttoc{toe}%
}


%%%%%%%%%%%%%%%%%%%%%%%%%%%%%%%%%%%%%%%%%%%%%%%%%%%%%%%%%%%%%%%%%%%%%%%%%%%%%%%%
% 封面摘要
%%%%%%%%%%%%%%%%%%%%%%%%%%%%%%%%%%%%%%%%%%%%%%%%%%%%%%%%%%%%%%%%%%%%%%%%%%%%%%%%
\def\cnauthorno#1{\def\@cnauthorno{#1}}\def\@cnauthorno{}

\def\cntitle#1{\def\@cntitle{#1}}\def\@cntitle{}
\def\cnaffil#1{\def\@cnaffil{#1}}\def\@cnaffil{}
\def\cnsubject#1{\def\@cnsubject{#1}}\def\@cnsubject{}
\def\cnauthor#1{\def\@cnauthor{#1}}\def\@cnauthor{}
\def\cnsupervisor#1{\def\@cnsupervisor{#1}}\def\@cnsupervisor{}
\def\cnsupervisortitle#1{\def\@cnsupervisortitle{#1}}\def\@cnsupervisortitle{}
\def\cnsubdate#1{\def\@cnsubdate{#1}}\def\@cnsubdate{}
\def\cndefdate#1{\def\@cndefdate{#1}}\def\@cndefdate{}
\long\def\cnabstract#1{\long\def\@cnabstract{#1}}\long\def\@cnabstract{}
\def\cnkeywords#1{\def\@cnkeywords{#1}}\def\@cnkeywords{}
%\def\cnreviewer#1{\def\@cnreviewer{#1}}\def\@cnreviewer{}

\def\entitle#1{\def\@entitle{#1}}\def\@entitle{}
\def\enaffil#1{\def\@enaffil{#1}}\def\@enaffil{}
\def\ensubject#1{\def\@ensubject{#1}}\def\@ensubject{}
\def\enauthor#1{\def\@enauthor{#1}}\def\@enauthor{}
\def\ensupervisor#1{\def\@ensupervisor{#1}}\def\@ensupervisor{}
\def\ensupervisortitle#1{\def\@ensupervisortitle{#1}}\def\@ensupervisortitle{}
\def\ensubdate#1{\def\@ensubdate{#1}}\def\@ensubdate{}
\def\endefdate#1{\def\@endefdate{#1}}\def\@endefdate{}
\long\def\enabstract#1{\long\def\@enabstract{#1}}\long\def\@enabstract{}
\def\enkeywords#1{\def\@enkeywords{#1}}\def\@enkeywords{}
\def\enreviewer#1{\def\@enreviewer{#1}}\def\@enreviewer{}

\long\def\NotationList#1{\long\def\@NotationList{#1}}\long\def\@NotationList{}
\long\def\cnauthorization#1{\long\def\@cnauthorization{#1}}\long\def\@cnauthorization{}
\long\def\cninnovation#1{\long\def\@cninnovation{#1}}\long\def\@cninnovation{}

\def\natclassifiedindex#1{\def\@natclassifiedindex{#1}}\def\@natclassifiedindex{}
\def\internatclassifiedindex#1{\def\@internatclassifiedindex{#1}}\def\@internatclassifiedindex{}

\def\studentno#1{\def\@studentno{#1}}\def\@studentno{}

\def\cnstatesecrets#1{\def\@cnstatesecrets{#1}}\def\@cnstatesecrets{}
\def\enstatesecrets#1{\def\@enstatesecrets{#1}}\def\@enstatesecrets{}

\def\cnuniversty#1{\def\@cnuniversty{#1}}\def\@cnuniversty{}
\def\enuniversty#1{\def\@euniversity{#1}}\def\@euniversity{}

% 封面
\def\makecover{
  \begin{titlepage}
    %内封(扉页)
    \newpage
    \thispagestyle{empty}
    \pdfbookmark[0]{\@cntitle}{cntitlepage}
    \begin{center}
        	\renewcommand{\arraystretch}{1.5}
        	{\song \sihao
        		\begin{tabular}{@{}r@{:}l@{}}
        			分类号              & \underline{\makebox[0.35\linewidth][c]{\@natclassifiedindex}}
        	\end{tabular}}\hfill
        	{\song \sihao
        		\begin{tabular}{@{}r@{:}l@{}}
        			密 \ 级 & \underline{\makebox[0.35\linewidth][c]{}}
        	\end{tabular}}
        	
        	{\song \sihao
        		\begin{tabular}{@{}r@{:}l@{}}
        			U \hfill D \hfill C& \underline{\makebox[0.88\linewidth][c]{}} 
        	\end{tabular}}  
        	
        	\renewcommand{\arraystretch}{1}
        	
        	\vspace*{12mm}
        	
        	\centerline{\song\erhao{华~~东~~理~~工~~大~~学}}
        	\vspace*{5mm}
        	\centerline{\song\erhao{学~~位~~论~~文}}
        	%                \vspace*{2mm}
        	
        	\underline{\parbox[t][10mm][t]{.6\textwidth}{
        			\begin{center}\sihao\song{\textbf{\@cntitle}}\end{center}}}
        	
        	%	            \vspace{5mm}
        	\underline{\parbox[t][12mm][t]{.6\textwidth}{
        			\begin{center}\sihao\song{\@cnauthor}\end{center}}}
        	
        	\vspace*{4mm}
        	\hspace*{-7em} 指导教师姓名:\underline{\parbox[c][6mm][c]{.6\textwidth}{
        			\begin{center}\sihao\song{\@cnsupervisor}~~华东理工大学\end{center}}}
        	
        	\vspace*{3mm}
        	\underline{\parbox[c][2mm][c]{.6\textwidth}{
        			\begin{center}\sihao\song{}\end{center}}}            	
        	
        	\vspace*{4mm}
        	%				\hspace*{-7em}
        	申请学位级别:\underline{\makebox[7em][c]{\cnxuewei}}\hfill
        	专~~~业~~名~~~称:\underline{\makebox[7em][c]{\@cnsubject}}
        	
        	论文定稿日期:\underline{\makebox[7em][c]{2022.04.11}}\hfill
        	论文答辩日期:\underline{\makebox[7em][c]{2022.05.01}}
        	
        	\hspace*{-1.5em}学位授予单位:\underline{\makebox[22em][c]{华东理工大学}}
        	
        	\hspace*{-1.5em}学位授予日期:\underline{\makebox[22em][c]{}}
        	
        	\vspace*{4em}
        	
        	\hspace*{6em}
        	\parbox[r][30mm][c]{\textwidth}
        	{
        		\begin{center}
        			\begin{tabular}{ll}
        				答辩委员会主席:& 周扒皮~~~~~ 教~~授\\
        				评\hfill 阅\hfill 人:&周扒皮 ~~~~~教~~授\\
        				&周扒皮 ~~~~~教~~授\\
        				&周扒皮 ~~~~~教~~授\\
        				&周扒皮 ~~~~~教~~授\\
        				&周扒皮 ~~~~~教~~授\\                			
        			\end{tabular}
        		\end{center}
        	}                
      \end{center}

    % 双面打印时封面后加空白页
%    \ifoneortwosidetwoside
%      \newpage
%      ~~~\vspace{1em}
%      \thispagestyle{empty}
%    \fi

    \cleardoublepage
  \end{titlepage}
}

% 原创性与使用授权说明
\def\authorization
{
    \thispagestyle{empty}
    \@cnauthorization
    \clearpage

%    % 双面打印时加空白页
%    \ifoneortwosidetwoside
%      \newpage
%      ~~~\vspace{1em}
%      \thispagestyle{empty}
%    \fi
}

\def\innovation
{
	\thispagestyle{empty}
	\@cninnovation
	\clearpage
	
%	% 双面打印时加空白页
%	\ifoneortwosidetwoside
%	\newpage
%	~~~\vspace{1em}
%	\thispagestyle{empty}
%	\fi
}


\def\makeabstract{
%	\setcounter{page}{1}
%	\vspace*{-5em}
	\defaultfont
\begin{center}
\parbox[c][20mm][c]{\linewidth}{
	\begin{center}\sihao\hei{\@cntitle}\end{center} }  

\parbox[c][20mm][c]{\linewidth}{
	\begin{center}\sihao\song{\textbf{摘~要}}\end{center} } 
\end{center}

  {\xiaosi \song \@cnabstract}
  \vspace{5mm}

  \noindent {\hei{关键词:{\fs\@cnkeywords}}}
  \setcounter{page}{1}
  
  \defaultfont
  \cleardoublepage
  
\begin{center}
	\parbox[c][20mm][c]{\textwidth}{
		\begin{center}\sihao\hei{\textbf{\@entitle}}\end{center} }  
	
	\parbox[c][20mm][c]{\textwidth}{
		\begin{center}\sihao\song{\textbf{Abstract}}\end{center} }  
\end{center}
	\vspace{-3mm}
	
  {\xiaosi \@enabstract}
  \vspace{5mm}

  \noindent {\textbf{Key Words:}}~~{\textsf{\@enkeywords}}
  \defaultfont
  \cleardoublepage
}

\makeatletter
\def\hlinewd#1{%
  \noalign{\ifnum0=`}\fi\hrule \@height #1 \futurelet
  \reserved@a\@xhline}
\makeatother

% 定义索引生成
\def\generateindex
{
  \addcontentsline{toc}{chapter}{\indexname}
  \printindex
  \cleardoublepage
}

\raggedbottom 