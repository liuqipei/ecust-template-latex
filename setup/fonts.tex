% !TEX TS-program = XeLaTeX
% !TEX encoding = UTF-8 Unicode

%%%%%%%%%%%%%%%%%%%%%%%%%%%%%%%%%%%%%%%%%%%%%%%%%%%%%%%%%%%%%%%%%%%%%%
%
%  华东理工大学本硕博论文 XeLaTeX 模版 —— 主文件 main.tex
%
%  版本:1.0.0
%  最后更新:
%  修改者:Liu Qipei, Wu Xinying, liuqipei@hrbeu.edu.cn, y10170069@mail.ecust.edu.cn
%  修订者:
%  编译环境1:macOS 10.13  + TeXLive 2020
%  编译环境2:Windows 10  + TeXLive 2020
%
%%%%%%%%%%%%%%%%%%%%%%%%%%%%%%%%%%%%%%%%%%%%%%%%%%%%%%%%%%%%%%%%%%%%%%

\usepackage{fontspec}
\usepackage{xltxtra,xunicode}
\usepackage[CJKnumber,CJKchecksingle,slantfont,boldfont]{xeCJK} % 允许斜体和粗体
\usepackage{amsmath}
\usepackage{amssymb}
\usepackage{bm}

\newif\iffontselectadobefont
\newif\iffontselectwindowsfont
\newif\iffontselectlinuxfont

\def\temp{adobefont}
\ifx\temp\fontselect
  \fontselectadobefonttrue  \fontselectwindowsfontfalse  \fontselectlinuxfontfalse
\fi

\def\temp{windowsfont}
\ifx\temp\fontselect
  \fontselectadobefontfalse  \fontselectwindowsfonttrue  \fontselectlinuxfontfalse
\fi

\def\temp{linuxfont}
\ifx\temp\fontselect
  \fontselectadobefontfalse  \fontselectwindowsfontfalse  \fontselectlinuxfonttrue
\fi

\iffontselectadobefont
% 字体设置特别推荐方案,需要安装 Adobe字体
% 英文字体设置
\setmainfont[Mapping=tex-text]{Times New Roman} %衬线字体
\setsansfont[Mapping=tex-text]{Arial}           %无衬线字体
\setmonofont[Mapping=tex-text]{Consolas}        %等宽字体

% 中文字体设置,使用的是 Adobe 字体,保证了在 Adobe Reader / Acrobat 下优秀的显示效果
\setCJKmainfont[BoldFont={AdobeHeitiStd-Regular}, ItalicFont={AdobeKaitiStd-Regular}]{AdobeSongStd-Light}
\setCJKsansfont{AdobeHeitiStd-Regular}
\setCJKmonofont{AdobeFangsongStd-Regular}

% 定义字体名称,可在此添加自定义的字体
\setCJKfamilyfont{song}{AdobeSongStd-Light}
\setCJKfamilyfont{hei}{AdobeHeitiStd-Regular}
\setCJKfamilyfont{kai}{AdobeKaitiStd-Regular}
\setCJKfamilyfont{fs}{AdobeFangsongStd-Regular}
%\setCJKfamilyfont{xkai}{STXingkai}
\fi

\iffontselectwindowsfont
    % 英文字体设置
    \setmainfont[Mapping=tex-text]{Times New Roman} %衬线字体
    \setsansfont[Mapping=tex-text]{Arial}           %无衬线字体
    \setmonofont[Mapping=tex-text]{Courier New}     %等宽字体

    % 中文字体设置,使用的是 Windows 系统字体
	\setCJKmainfont[BoldFont={SimHei}, ItalicFont={KaiTi}]{NSimSun}
	\setCJKsansfont{SimHei}
	\setCJKmonofont{FangSong}

	\setCJKfamilyfont{song}{NSimSun}
	\setCJKfamilyfont{hei}{SimHei}
	\setCJKfamilyfont{kai}{KaiTi}   % XP对应 KaiTi_GB2312,Vista对应KaiTi,注意根据系统切换
	\setCJKfamilyfont{fs}{FangSong} % XP对应 FangSong_GB2312,Vista对应FangSong,注意根据系统切换
\fi

\iffontselectlinuxfont
    % 经典 Linux 英文字体设置方案
    \setmainfont[Mapping=tex-text]{LMRoman10} %衬线字体
    \setsansfont[Mapping=tex-text]{LMSans10}  %无衬线字体
    \setmonofont[Mapping=tex-text]{LMMono10}  %等宽字体

    % 中文字体设置,使用的是 Adobe 字体,保证了在 Adobe Reader / Acrobat 下优秀的显示效果
    \setCJKmainfont[BoldFont={Adobe Heiti Std}, ItalicFont={Adobe Kaiti Std}]{Adobe Song Std}
    \setCJKsansfont{Adobe Heiti Std}
    \setCJKmonofont{Adobe Fangsong Std}

    % 定义字体名称,可在此添加自定义的字体
    \setCJKfamilyfont{song}{Adobe Song Std}
    \setCJKfamilyfont{hei}{Adobe Heiti Std}
    \setCJKfamilyfont{kai}{Adobe Kaiti Std}
    \setCJKfamilyfont{fs}{Adobe Fangsong Std}
    %\setCJKfamilyfont{xkai}{STXingkai}
\fi

\newcommand{\song}{\CJKfamily{song}}
\newcommand{\hei}{\CJKfamily{hei}}
\newcommand{\kai}{\CJKfamily{kai}}
\newcommand{\fs}{\CJKfamily{fs}}

% 定义CJK兼容的汉字字体别名
\def\songti{\song}
\def\fangsong{\fs}
\def\kaishu{\kai}
\def\heiti{\hei}

% 字号
\newcommand{\chuhao}{\fontsize{42pt}{50.5pt}\selectfont}    % 初号,1.25  倍行距
\newcommand{\xiaochu}{\fontsize{36pt}{45pt}\selectfont}     % 小初,1.25  倍行距
\newcommand{\yihao}{\fontsize{26pt}{39pt}\selectfont}       % 一号,1.5  倍行距
\newcommand{\xiaoyi}{\fontsize{24pt}{30pt}\selectfont}      % 小一,1.25 倍行距
\newcommand{\erhao}{\fontsize{22pt}{27.5pt}\selectfont}     % 二号,1.25 倍行距
\newcommand{\xiaoer}{\fontsize{18pt}{22.5pt}\selectfont}    % 小二,1.25 倍行距
\newcommand{\sanhao}{\fontsize{16pt}{20pt}\selectfont}      % 三号,1.25 倍行距
\newcommand{\xiaosan}{\fontsize{15pt}{19pt}\selectfont}     % 小三,1.25 倍行距
\newcommand{\sihao}{\fontsize{14pt}{17.5pt}\selectfont}     % 四号,1.25倍行距
\newcommand{\daxiaosi}{\fontsize{12pt}{18pt}\selectfont}    % 小四,1.5 倍行距
\newcommand{\xiaosi}{\fontsize{12pt}{15pt}\selectfont}      % 小四,1.25倍行距
\newcommand{\dawu}{\fontsize{10.5pt}{18pt}\selectfont}      % 五号,1.75倍行距
\newcommand{\zhongwu}{\fontsize{10.5pt}{16pt}\selectfont}   % 五号,1.5 倍行距
\newcommand{\wuhao}{\fontsize{10.5pt}{10.5pt}\selectfont}   % 五号,单倍行距
\newcommand{\xiaowu}{\fontsize{9pt}{9pt}\selectfont}        % 小五,单倍行距

% 自动调整中英文之间的空白
% \punctstyle{quanjiao}
\XeTeXlinebreaklocale "zh"      %中文断行
\XeTeXlinebreakskip = 0pt plus 1pt %1pt左右弹性间距
% 其他字体宏包
