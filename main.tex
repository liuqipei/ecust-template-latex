% !TEX TS-program = XeLaTeX
% !TEX encoding = UTF-8 Unicode

%%%%%%%%%%%%%%%%%%%%%%%%%%%%%%%%%%%%%%%%%%%%%%%%%%%%%%%%%%%%%%%%%%%%%%
%
%  华东理工大学本硕博论文 XeLaTeX 模版 —— 主文件 main.tex
%
%  版本:1.0.0
%  最后更新:
%  修改者:Liu Qipei, Wu Xinying, liuqipei@hrbeu.edu.cn, y10170069@mail.ecust.edu.cn
%  修订者:
%  编译环境1:macOS 10.13  + TeXLive 2020
%  编译环境2:Windows 10  + TeXLive 2020
%
%%%%%%%%%%%%%%%%%%%%%%%%%%%%%%%%%%%%%%%%%%%%%%%%%%%%%%%%%%%%%%%%%%%%%%

%%----------  以下是学位、学科类型与打印方式选择 ---------------------
%%只对博士论文模板进行了修订,其余类型请使用者参照博士模板进行修改即可
%%----  1.选择学位论文类型,可以是:
%%   Doctor    -博士
%%   Master    -硕士
%%   Bachelor  -学士
\def\xuewei{Doctor}

%%  2.定义学科,可以是:
%%  Engineering   -工学
%%  Science       -科学
%%  Management    -管理
%%  Arts          -艺术
%%  Philosophy    -哲学
%%  Economics     -经济
%%  Laws          -法律
%%  Education     -教育
%%  History       -历史
\def\xueke{Engineering}

%%  3.选择字体,可以是:
%%   adobefont     -Adobe 汉字库
%%   windowsfont   -Windows系统汉字库
%%   linuxfont     -Linux 系统字库
%% 	 macOSfont     -macOS 系统字库
\def\fontselect{adobefont}

%%  4.选择打印单双面打印方式,可以是:
%%   oneside     -单面打印
%%   twoside     -双面打印
\def\oneortwoside{twoside}


%%----------  论文基本格式设置  -------------------------------------

\documentclass[12pt, a4paper, openany, \oneortwoside]{book}

% 字体配置文件
% !TEX TS-program = XeLaTeX
% !TEX encoding = UTF-8 Unicode

%%%%%%%%%%%%%%%%%%%%%%%%%%%%%%%%%%%%%%%%%%%%%%%%%%%%%%%%%%%%%%%%%%%%%%
%
%  华东理工大学本硕博论文 XeLaTeX 模版 —— 主文件 main.tex
%
%  版本:1.0.0
%  最后更新:
%  修改者:Liu Qipei, Wu Xinying, liuqipei@hrbeu.edu.cn, y10170069@mail.ecust.edu.cn
%  修订者:
%  编译环境1:macOS 10.13  + TeXLive 2020
%  编译环境2:Windows 10  + TeXLive 2020
%
%%%%%%%%%%%%%%%%%%%%%%%%%%%%%%%%%%%%%%%%%%%%%%%%%%%%%%%%%%%%%%%%%%%%%%

\usepackage{fontspec}
\usepackage{xltxtra,xunicode}
\usepackage[CJKnumber,CJKchecksingle,slantfont,boldfont]{xeCJK} % 允许斜体和粗体
\usepackage{amsmath}
\usepackage{amssymb}
\usepackage{bm}

\newif\iffontselectadobefont
\newif\iffontselectwindowsfont
\newif\iffontselectlinuxfont

\def\temp{adobefont}
\ifx\temp\fontselect
  \fontselectadobefonttrue  \fontselectwindowsfontfalse  \fontselectlinuxfontfalse
\fi

\def\temp{windowsfont}
\ifx\temp\fontselect
  \fontselectadobefontfalse  \fontselectwindowsfonttrue  \fontselectlinuxfontfalse
\fi

\def\temp{linuxfont}
\ifx\temp\fontselect
  \fontselectadobefontfalse  \fontselectwindowsfontfalse  \fontselectlinuxfonttrue
\fi

\iffontselectadobefont
% 字体设置特别推荐方案,需要安装 Adobe字体
% 英文字体设置
\setmainfont[Mapping=tex-text]{Times New Roman} %衬线字体
\setsansfont[Mapping=tex-text]{Arial}           %无衬线字体
\setmonofont[Mapping=tex-text]{Consolas}        %等宽字体

% 中文字体设置,使用的是 Adobe 字体,保证了在 Adobe Reader / Acrobat 下优秀的显示效果
\setCJKmainfont[BoldFont={AdobeHeitiStd-Regular}, ItalicFont={AdobeKaitiStd-Regular}]{AdobeSongStd-Light}
\setCJKsansfont{AdobeHeitiStd-Regular}
\setCJKmonofont{AdobeFangsongStd-Regular}

% 定义字体名称,可在此添加自定义的字体
\setCJKfamilyfont{song}{AdobeSongStd-Light}
\setCJKfamilyfont{hei}{AdobeHeitiStd-Regular}
\setCJKfamilyfont{kai}{AdobeKaitiStd-Regular}
\setCJKfamilyfont{fs}{AdobeFangsongStd-Regular}
%\setCJKfamilyfont{xkai}{STXingkai}
\fi

\iffontselectwindowsfont
    % 英文字体设置
    \setmainfont[Mapping=tex-text]{Times New Roman} %衬线字体
    \setsansfont[Mapping=tex-text]{Arial}           %无衬线字体
    \setmonofont[Mapping=tex-text]{Courier New}     %等宽字体

    % 中文字体设置,使用的是 Windows 系统字体
	\setCJKmainfont[BoldFont={SimHei}, ItalicFont={KaiTi}]{NSimSun}
	\setCJKsansfont{SimHei}
	\setCJKmonofont{FangSong}

	\setCJKfamilyfont{song}{NSimSun}
	\setCJKfamilyfont{hei}{SimHei}
	\setCJKfamilyfont{kai}{KaiTi}   % XP对应 KaiTi_GB2312,Vista对应KaiTi,注意根据系统切换
	\setCJKfamilyfont{fs}{FangSong} % XP对应 FangSong_GB2312,Vista对应FangSong,注意根据系统切换
\fi

\iffontselectlinuxfont
    % 经典 Linux 英文字体设置方案
    \setmainfont[Mapping=tex-text]{LMRoman10} %衬线字体
    \setsansfont[Mapping=tex-text]{LMSans10}  %无衬线字体
    \setmonofont[Mapping=tex-text]{LMMono10}  %等宽字体

    % 中文字体设置,使用的是 Adobe 字体,保证了在 Adobe Reader / Acrobat 下优秀的显示效果
    \setCJKmainfont[BoldFont={Adobe Heiti Std}, ItalicFont={Adobe Kaiti Std}]{Adobe Song Std}
    \setCJKsansfont{Adobe Heiti Std}
    \setCJKmonofont{Adobe Fangsong Std}

    % 定义字体名称,可在此添加自定义的字体
    \setCJKfamilyfont{song}{Adobe Song Std}
    \setCJKfamilyfont{hei}{Adobe Heiti Std}
    \setCJKfamilyfont{kai}{Adobe Kaiti Std}
    \setCJKfamilyfont{fs}{Adobe Fangsong Std}
    %\setCJKfamilyfont{xkai}{STXingkai}
\fi

\newcommand{\song}{\CJKfamily{song}}
\newcommand{\hei}{\CJKfamily{hei}}
\newcommand{\kai}{\CJKfamily{kai}}
\newcommand{\fs}{\CJKfamily{fs}}

% 定义CJK兼容的汉字字体别名
\def\songti{\song}
\def\fangsong{\fs}
\def\kaishu{\kai}
\def\heiti{\hei}

% 字号
\newcommand{\chuhao}{\fontsize{42pt}{50.5pt}\selectfont}    % 初号,1.25  倍行距
\newcommand{\xiaochu}{\fontsize{36pt}{45pt}\selectfont}     % 小初,1.25  倍行距
\newcommand{\yihao}{\fontsize{26pt}{39pt}\selectfont}       % 一号,1.5  倍行距
\newcommand{\xiaoyi}{\fontsize{24pt}{30pt}\selectfont}      % 小一,1.25 倍行距
\newcommand{\erhao}{\fontsize{22pt}{27.5pt}\selectfont}     % 二号,1.25 倍行距
\newcommand{\xiaoer}{\fontsize{18pt}{22.5pt}\selectfont}    % 小二,1.25 倍行距
\newcommand{\sanhao}{\fontsize{16pt}{20pt}\selectfont}      % 三号,1.25 倍行距
\newcommand{\xiaosan}{\fontsize{15pt}{19pt}\selectfont}     % 小三,1.25 倍行距
\newcommand{\sihao}{\fontsize{14pt}{17.5pt}\selectfont}     % 四号,1.25倍行距
\newcommand{\daxiaosi}{\fontsize{12pt}{18pt}\selectfont}    % 小四,1.5 倍行距
\newcommand{\xiaosi}{\fontsize{12pt}{15pt}\selectfont}      % 小四,1.25倍行距
\newcommand{\dawu}{\fontsize{10.5pt}{18pt}\selectfont}      % 五号,1.75倍行距
\newcommand{\zhongwu}{\fontsize{10.5pt}{16pt}\selectfont}   % 五号,1.5 倍行距
\newcommand{\wuhao}{\fontsize{10.5pt}{10.5pt}\selectfont}   % 五号,单倍行距
\newcommand{\xiaowu}{\fontsize{9pt}{9pt}\selectfont}        % 小五,单倍行距

% 自动调整中英文之间的空白
% \punctstyle{quanjiao}
\XeTeXlinebreaklocale "zh"      %中文断行
\XeTeXlinebreakskip = 0pt plus 1pt %1pt左右弹性间距
% 其他字体宏包

% 宏包配置文件
% !TEX TS-program = XeLaTeX
% !TEX encoding = UTF-8 Unicode

%%%%%%%%%%%%%%%%%%%%%%%%%%%%%%%%%%%%%%%%%%%%%%%%%%%%%%%%%%%%%%%%%%%%%%
%
%  华东理工大学本硕博论文 XeLaTeX 模版 —— 主文件 main.tex
%
%  版本:1.0.0
%  最后更新:
%  修改者:Liu Qipei, Wu Xinying, liuqipei@hrbeu.edu.cn, y10170069@mail.ecust.edu.cn
%  修订者:
%  编译环境1:macOS 10.13  + TeXLive 2020
%  编译环境2:Windows 10  + TeXLive 2020
%
%%%%%%%%%%%%%%%%%%%%%%%%%%%%%%%%%%%%%%%%%%%%%%%%%%%%%%%%%%%%%%%%%%%%%%

% 页面设置
\usepackage{geometry}
\usepackage{indentfirst}                         % 首行缩进宏包
\usepackage[center]{titlesec}                    % 控制标题的宏包
\usepackage{titletoc}                            % 控制目录的宏包
\usepackage{fancyhdr}                            % 自定义页眉页脚
\usepackage{fancyref}                            % 引用链接属性
\usepackage[perpage,symbol]{footmisc}            % 脚注控制
\usepackage{layouts}                             % 打印当前页面格式的宏包
\usepackage{paralist}                            % 一种换行不缩进的列表格式,asparaenum,inparaenum 等
\usepackage[shortlabels]{enumitem}               % 列表格式
\usepackage{fancyvrb}                            % 原样输出
\usepackage[amsmath,thmmarks,hyperref]{ntheorem} % 定理类环境宏包
\usepackage{type1cm}                             % 控制字体的大小

% 表格处理
\usepackage{booktabs}   % 三线表
\usepackage{multirow}   % 表格多行处理
\usepackage{diagbox}    % 斜线表头
\usepackage{tabularx}   % 表格折行
\usepackage{siunitx}    % 国际单位,小数点对齐


% 图形相关
\usepackage{graphicx}          % 请在引用图片时务必给出后缀名
\usepackage[x11names]{xcolor}  % 支持彩色
\usepackage[below]{placeins}   % 浮动图形控制宏包
\usepackage{rotating}          % 图形和表格的控制
\usepackage{picinpar}
\usepackage{setspace}          % 定制表格和图形的多行标题行距
\usepackage{subfigure}           % 插入子图形
\usepackage[subfigure]{ccaption} % 插图表格的双语标题
\usepackage{tikz}
\usepackage{pifont}           % 带圈数字①-⑩

% 其他
\usepackage{calc}   % 在 tex 文件中具有一些计算功能,主要用在页面控制。

%\usepackage[numbers,sort&compress,square,super]{natbib} %参考文献
\usepackage[numbers,sort&compress,square]{natbib} %参考文献

\usepackage{hypernat}
\usepackage{bibentry}

\usepackage{listings}         % 源代码展示
\lstset{%
  language=TeX,
  defaultdialect=empty,
  basicstyle=\ttfamily\small,
  backgroundcolor=\color{LightSteelBlue1},
  keywordstyle=\color{blue},
  showspaces=false,
  showstringspaces=false,
  showtabs=false,
  tabsize=2,breakatwhitespace=false,
  columns=flexible}

% 生成有书签的pdf及其开关, 该宏包应放在所有宏包的最后, 宏包之间有冲突
\usepackage[xetex,
            bookmarksnumbered=true,
            bookmarksopen=true,
            colorlinks=true,
            %pdfborder={0 0 1},
            citecolor=blue, % 文献标柱颜色
            linkcolor=black, % 锚点颜色
            anchorcolor=green, % 超链颜色
            urlcolor=blue,
            breaklinks=true,
			CJKbookmarks=true,
            naturalnames  %与algorithm2e宏包协调
            ]{hyperref}

% 格式文件
% !TEX TS-program = XeLaTeX
% !TEX encoding = UTF-8 Unicode

%%%%%%%%%%%%%%%%%%%%%%%%%%%%%%%%%%%%%%%%%%%%%%%%%%%%%%%%%%%%%%%%%%%%%%
%
%  华东理工大学本硕博论文 XeLaTeX 模版 —— 主文件 main.tex
%
%  版本:1.0.0
%  最后更新:
%  修改者:Liu Qipei, Wu Xinying, liuqipei@hrbeu.edu.cn, y10170069@mail.ecust.edu.cn
%  修订者:
%  编译环境1:macOS 10.13  + TeXLive 2020
%  编译环境2:Windows 10  + TeXLive 2020
%
%%%%%%%%%%%%%%%%%%%%%%%%%%%%%%%%%%%%%%%%%%%%%%%%%%%%%%%%%%%%%%%%%%%%%%

%%%%%%%%%%%%%%%%%%%%%%%%%%%%%%%%%%%%%%%%%%%%%%%%%%%%%%%%%%%%%%%%%%%%%%
% 页面设置
%%%%%%%%%%%%%%%%%%%%%%%%%%%%%%%%%%%%%%%%%%%%%%%%%%%%%%%%%%%%%%%%%%%%%%
% A4 纸张
\setlength{\paperwidth}{210mm}
\setlength{\paperheight}{297mm}

% 设置正文尺寸大小
\setlength{\textwidth}{160mm}
\setlength{\textheight}{245mm}

% 设置正文区在正中间
\newlength \mymargin
\setlength{\mymargin}{(\paperwidth-\textwidth)/2}
\setlength{\oddsidemargin}{(\mymargin)-1in}
\setlength{\evensidemargin}{(\mymargin)-1in}

% 设置正文区偏移量,奇数页向右偏,偶数页向左偏
\newlength \myshift
\setlength{\myshift}{0mm}    % 双面打印的奇偶页偏移值,可根据需要修改,建议小于 5mm
\addtolength{\oddsidemargin}{\myshift}
\addtolength{\evensidemargin}{-\myshift}

% 页眉页脚相关距离设置
\setlength{\voffset}{-5.4mm} % 设置水平基线位置
\setlength{\topmargin}{0mm}  % 设置页眉距水平基线位置
\setlength{\headheight}{5mm} % 设置页眉高度
\setlength{\headsep}{8mm}    % 页眉与正文的距离
\setlength{\footskip}{6mm}   % 页脚与正文距离

% 公式的精调
\allowdisplaybreaks[4]  % 可以让公式在排不下的时候分页排,这可避免页面有大段空白。

% 下面这组命令使浮动对象的缺省值稍微宽松一点,从而防止幅度
% 对象占据过多的文本页面,也可以防止在很大空白的浮动页上放置很小的图形。
\renewcommand{\topfraction}{0.9999999}
\renewcommand{\textfraction}{0.0000001}
\renewcommand{\floatpagefraction}{0.9999}

% defaultfont 默认字体命令
\def\defaultfont{\renewcommand{\baselinestretch}{1.25} \daxiaosi}
%  \fontsize{12pt}{15pt}\selectfont}

% 设置目录字体和行间距
\def\defaultmenufont{\renewcommand{\baselinestretch}{1.22} \xiaosi}
%  \fontsize{12pt}{15pt}\selectfont}

% 固定距离内容填入及下划线
\makeatletter
\newcommand\fixeddistanceleft[2][10mm]{{\hb@xt@ #1{#2\hss}}}
\newcommand\fixeddistancecenter[2][10mm]{{\hb@xt@ #1{\hss#2\hss}}}
\newcommand\fixeddistanceright[2][10mm]{{\hb@xt@ #1{\hss#2}}}
\newcommand\fixedunderlineleft[2][10mm]{\underline{\hb@xt@ #1{#2\hss}}}
\newcommand\fixedunderlinecenter[2][10mm]{\underline{\hb@xt@ #1{\hss#2\hss}}}
\newcommand\fixedunderlineright[2][10mm]{\underline{\hb@xt@ #1{\hss#2}}}
\makeatother

%%%%%%%%%%%%%%%%%%%%%%%%%%%%%%%%%%%%%%%%%%%%%%%%%%%%%%%%%%%%%%%%%%%%%%
% 标题环境相关
%%%%%%%%%%%%%%%%%%%%%%%%%%%%%%%%%%%%%%%%%%%%%%%%%%%%%%%%%%%%%%%%%%%%%%
%判断单双面打印类型
\newif\ifoneortwosidetwoside
\newif\ifoneortwosideoneside

\def\temp{twoside}
\ifx\temp\oneortwoside
  \oneortwosidetwosidetrue  \oneortwosideonesidefalse
\fi

\def\temp{oneside}
\ifx\temp\oneortwoside
  \oneortwosidetwosidefalse  \oneortwosideonesidetrue
\fi

%判断论文类型
% 声明三个论文类型逻辑型变量
\newif\ifxueweidoctor
\newif\ifxueweimaster
\newif\ifxueweibachelor

% 根据 \xuewei 的定义为 \xueweidoctor \xueweimaster \xueweibachelor 赋值
\def\temp{Doctor}
\ifx\temp\xuewei % \ifx 用于判断两个变量是否匹配
  \xueweidoctortrue  \xueweimasterfalse \xueweibachelorfalse
\fi
\def\temp{Master}
\ifx\temp\xuewei
  \xueweidoctorfalse  \xueweimastertrue \xueweibachelorfalse
\fi
\def\temp{Bachelor}
\ifx\temp\xuewei
  \xueweidoctorfalse  \xueweimasterfalse \xueweibachelortrue
\fi

\ifxueweidoctor
  \newcommand{\cnxuewei}{博士}
  \newcommand{\enxuewei}{Doctor}
\fi

\ifxueweimaster
  \newcommand{\cnxuewei}{硕士}
  \newcommand{\enxuewei}{Master}
\fi

\ifxueweibachelor
  \newcommand{\cnxuewei}{学士}
  \newcommand{\enxuewei}{Bachelor}
\fi

%定义 学科 学位
\def \xuekeEngineering {Engineering}
\def \xuekeScience     {Science}
\def \xuekeManagement  {Management}
\def \xuekeArts        {Arts}
\def \xuekePhilosophy  {Philosophy}
\def \xuekeEconomics   {Economics}
\def \xuekeLaws        {Laws}
\def \xuekeEducation   {Education}
\def \xuekeHistory     {History}


\ifx \xueke \xuekeEngineering
\newcommand{\cnxueke}{工学}
\newcommand{\enxueke}{Engineering}
\newcommand{\enxk}   {Eng}
\fi

\ifx \xueke \xuekeScience
\newcommand{\cnxueke}{理学}
\newcommand{\enxueke}{Science}
\newcommand{\enxk}   {Sci}
\fi

\ifx \xueke \xuekeManagement
\newcommand{\cnxueke}{管理学}
\newcommand{\enxueke}{Management}
\newcommand{\enxk}   {Man}
\fi

\ifx \xueke \xuekeArts
\newcommand{\cnxueke}{文学}
\newcommand{\enxueke}{Arts}
\newcommand{\enxk}   {Art}
\fi

\ifx \xueke \xuekePhilosophy
\newcommand{\cnxueke}{哲学}
\newcommand{\enxueke}{Philosophy}
\newcommand{\enxk}   {Phi}
\fi

\ifx \xueke \xuekeEconomics
\newcommand{\cnxueke}{经济学}
\newcommand{\enxueke}{Economics}
\newcommand{\enxk}   {Eco}
\fi

\ifx \xueke \xuekeLaws
\newcommand{\cnxueke}{法学}
\newcommand{\enxueke}{Laws}
\newcommand{\enxk}   {Law}
\fi

\ifx \xueke \xuekeEducation
\newcommand{\cnxueke}{教育学}
\newcommand{\enxueke}{Education}
\newcommand{\enxk}   {Edu}
\fi

\ifx \xueke \xuekeHistory
\newcommand{\cnxueke}{历史学}
\newcommand{\enxueke}{History}
\newcommand{\enxk}   {His}
\fi

% 定义、定理等环境
\theoremstyle{plain}
\theoremheaderfont{\hei\bf}
\theorembodyfont{\song\rmfamily}
\newtheorem{definition}{\hei 定义}[chapter]
\newtheorem{example}{\hei 例}[chapter]
\newtheorem{algorithm}{\hei 算法}[chapter]
\newtheorem{theorem}{\hei 定理}[chapter]
\newtheorem{axiom}{\hei 公理}[chapter]
\newtheorem{proposition}[theorem]{\hei 命题}
\newtheorem{property}{\hei 性质}
\newtheorem{lemma}[theorem]{\hei 引理}
\newtheorem{corollary}{\hei 推论}[chapter]
\newtheorem{remark}{\hei 注解}[chapter]
\newenvironment{proof}{\hei{证明} }{\hfill $\square$ \vskip 4mm}

% 目录标题
\renewcommand{\contentsname}{\hfill \hei 目~~~~录 \hfill}
\renewcommand{\listfigurename}{\hfill 插~图~目~录 \hfill}
\renewcommand{\listtablename}{\hfill 表~格~目~录 \hfill}
\renewcommand{\bibname}{\hfill 参~考~文~献 \hfill}
\renewcommand\appendixname{附~录}

%%%%%%%%%%%%%%%%%%%%%%%%%%%%%%%%%%%%%%%%%%%%%%%%%%%%%%%%%%%%%%%%%%%%%%
% 段落章节相关
%%%%%%%%%%%%%%%%%%%%%%%%%%%%%%%%%%%%%%%%%%%%%%%%%%%%%%%%%%%%%%%%%%%%%%
\setcounter{secnumdepth}{4}
\setcounter{tocdepth}{4}
\setcounter{chapter}{0}

% 设置章、节、小节、小小节的间距
%\titleformat{\chapter}[hang]{\normalfont\xiaosan\hei\sf}{\xiaosan\thechapter}{1em}{\xiaosan}
\titleformat{\chapter}{\centering\sihao\hei\sf}{第\,\thechapter\,章}{1em}{} % 设置中文章格式中央对齐:第  章
\titlespacing{\chapter}{0pt}{-3ex  plus .1ex minus .2ex}{3.3ex}
\titleformat{\section}[hang]{\xiaosi\song\sf}{\xiaosi\thesection}{1em}{}{}
\titlespacing{\section}{0pt}{0.5em}{0.5em}
\titleformat{\subsection}[hang]{\xiaosi\song}{\xiaosi\thesubsection}{1em}{}{}
\titlespacing{\subsection}{0pt}{0.5em}{0.3em}
\titleformat{\subsubsection}[hang]{\xiaosi\hei\sf}{\xiaosi\thesubsubsection}{1em}{}{}
\titlespacing{\subsubsection}{0pt}{0.3em}{0pt}

% 缩小目录中各级标题之间的缩进
% \dottedcontents{<section>}[<left>]{<above>}{<labelwidth>}{<leaderwidth>}
\dottedcontents{chapter}[3mm]{\vspace{0.2em}}{1.0em}{5pt}
\dottedcontents{section}[8mm]{}{1.8em}{5pt}
\dottedcontents{subsection}[23mm]{}{2.7em}{5pt}
\dottedcontents{subsubsection}[33mm]{}{3.4em}{5pt}

% 设置目录中各级标题之间的缩进
\makeatletter
\renewcommand*{\l@chapter}{\@dottedtocline{0}{0em}{1em}}% 细点\@dottedtocline  粗点\@dottedtoclinebold
\renewcommand*{\l@section}{\@dottedtocline{1}{0em}{2em}}
\renewcommand*{\l@subsection}{\@dottedtocline{2}{0em}{3em}}
\renewcommand*{\l@subsubsection}{\@dottedtocline{3}{0em}{4em}}

% 设置章标题格式
% \titlecontents{章节名称}[左端距离]{标题字体、与上文间距等}{标题序号}{空}{引导符和页码}[与下文间距]
% 设置目录的点间距
\titlecontents{chapter}[3.8em]{\hspace{-3.8em}\hei}{第~\thecontentslabel~章~~\hspace{.6em}}{}{\titlerule*[8pt]{.}\contentspage}

% 段落之间的竖直距离
\setlength{\parskip}{1.2pt}
% 段落缩进
\setlength{\parindent}{24pt}
% 定义行距
\renewcommand{\baselinestretch}{1.25}
% 参考文献条目间行间距
\setlength{\bibsep}{2pt}


%%%%%%%%%%%%%%%%%%%%%%%%%%%%%%%%%%%%%%%%%%%%%%%%%%%%%%%%%%%%%%%%%%%%%%
% 页眉页脚设置
%%%%%%%%%%%%%%%%%%%%%%%%%%%%%%%%%%%%%%%%%%%%%%%%%%%%%%%%%%%%%%%%%%%%%%

\newcommand{\makeheadrule}{%
    \rule[12pt]{\textwidth}{0.5pt} \\[-23pt]
%    \rule{\textwidth}{2.0pt}
  \vskip-.8\baselineskip}

\makeatletter
\renewcommand{\headrule}{%
  {\if@fancyplain\let\headrulewidth\plainheadrulewidth\fi
    \makeheadrule}}

\pagestyle{fancyplain}

%去掉章节标题中的数字
%%不要注销这一行,否则页眉会变成:“第1章1  绪论”样式
\renewcommand{\chaptermark}[1]{\markboth{第\thechapter 章~~~~\ #1}{}}
\fancyhf{}

% 附录设置:附录不编章节号,但列入目录和页眉
\renewcommand{\appendix}[1]{%
    \chapter*{#1}%
    \addcontentsline{toc}{chapter}{#1}%
    \markboth{#1}{#1}
}

%在book文件类别下,\leftmark自动存录各章之章名,\rightmark记录节标题
%根据单双面打印设置不同的页眉;
\fancyhead[CO]{\song\xiaosi{\sihao\kai\textbf{\@cnuniversty}}~~\cnxuewei 学位论文\hfill{第$~\thepage~$页}}
\fancyhead[CE]{\song\xiaosi{第$~\thepage~$页\hfill{\sihao\kai\textbf{\@cnuniversty}}~~\cnxuewei 学位论文}}
%\fancyfoot[C,C]{\xiaosi$-$~\thepage~$-$}

%偶数页为空白页面时处理
%\makeatletter
%\def\cleardoublepage{\clearpage\if@twoside \ifodd\c@page\else%
%  \hbox{}%
%%  \thispagestyle{empty}%   % 清除页眉、页脚
%  \vspace*{80mm}
%  \centerline{\xiaoer\song {{(此页无正文)}}}
%  \centerline{\xiaosi\song {{(THIS PAGE IS INTENTIONALLY LEFT BLANK)}}}
%  \newpage%
%  \if@twocolumn\hbox{}\newpage\fi\fi\fi}

\makeatletter
\def\cleardoublepage{\newpage}

%%%%%%%%%%%%%%%%%%%%%%%%%%%%%%%%%%%%%%%%%%%%%%%%%%%%%%%%%%%%%%%%%%%%%%
% 列表环境设置
%%%%%%%%%%%%%%%%%%%%%%%%%%%%%%%%%%%%%%%%%%%%%%%%%%%%%%%%%%%%%%%%%%%%%%

\setlist[enumerate]{(1),itemsep=-5pt,topsep=0mm,labelindent=\parindent,leftmargin=*}

%%%%%%%%%%%%%%%%%%%%%%%%%%%%%%%%%%%%%%%%%%%%%%%%%%%%%%%%%%%%%%%%%%%%%%
% 国际单位,以点连接。
%%%%%%%%%%%%%%%%%%%%%%%%%%%%%%%%%%%%%%%%%%%%%%%%%%%%%%%%%%%%%%%%%%%%%%
\sisetup{inter-unit-product = { }\cdot{ }}

%%%%%%%%%%%%%%%%%%%%%%%%%%%%%%%%%%%%%%%%%%%%%%%%%%%%%%%%%%%%%%%%%%%%%%
% 参考文献的处理
%%%%%%%%%%%%%%%%%%%%%%%%%%%%%%%%%%%%%%%%%%%%%%%%%%%%%%%%%%%%%%%%%%%%%%

% \addtolength{\bibsep}{-0.5 em}      % 缩小参考文献间的垂直间距
% 上标引用,比\cite位置更偏上、字号稍小
\DeclareRobustCommand\scite{\@scite}
\def\@scite#1{\textsuperscript{\cite{#1}}}
% 行间引用,与正文格式一致
\DeclareRobustCommand\lcite{\@lcite}
\def\@lcite#1{\begingroup\let\@cite\NAT@citenum\citep{#1}\endgroup}

\setlength{\bibhang}{2em}
\bibpunct{[}{]}{,}{s}{}{}


%%%%%%%%%%%%%%%%%%%%%%%%%%%%%%%%%%%%%%%%%%%%%%%%%%%%%%%%%%%%%%%%%%%%%%
%   其他设置
%%%%%%%%%%%%%%%%%%%%%%%%%%%%%%%%%%%%%%%%%%%%%%%%%%%%%%%%%%%%%%%%%%%%%%
% 使图编号为 7-1 的格式 %\protect{~}
\renewcommand{\thefigure}{\arabic{chapter}.\arabic{figure}}
% 使子图编号为 (a)的格式
\renewcommand{\thesubfigure}{(\alph{subfigure})}
% 使子图引用为 7-1 (a) 的格式,母图编号和子图编号之间用~加一个空格
\renewcommand{\p@subfigure}{\thefigure~}
% 使表编号为 7-1 的格式
\renewcommand{\thetable}{\arabic{chapter}.\arabic{table}}
% 使公式编号为 7-1 的格式
\renewcommand{\theequation}{\arabic{chapter}-\arabic{equation}}

%插图索引格式: 图 x. 图标题 ......页码
\renewcommand\listoffigures{%
    \chapter*{\listfigurename}%
    \addcontentsline{toc}{chapter}{插图目录}
    \markboth{\listfigurename}{\listfigurename}
    \renewcommand{\numberline}[1]{图~##1~~}
    \@starttoc{lof}%
    }

%表格索引格式: 表 x. 表标题 ......页码
\renewcommand\listoftables{%
    \chapter*{\listtablename}%
    \addcontentsline{toc}{chapter}{表格目录}
    \markboth{\listtablename}{\listtablename}
    \renewcommand{\numberline}[1]{表~##1~~}
    \@starttoc{lot}%
    }

%%%%%%%%%%%%%%%%%%%%%%%%%%%%%%%%%%%%%%%%%%%%%%%%%%%%%%%%%%%%%%%%%%%%%%
%   图形表格
%%%%%%%%%%%%%%%%%%%%%%%%%%%%%%%%%%%%%%%%%%%%%%%%%%%%%%%%%%%%%%%%%%%%%%
\renewcommand{\figurename}{\song\textbf{图}}
\renewcommand{\tablename}{\song\textbf{表}}
% \captionstyle{\centering}
% \hangcaption
\captiondelim{\hspace{1em}}
\captiondelim{\hspace{1em}}
\captionnamefont{\zhongwu\song}
\captiontitlefont{\zhongwu\song}
\setlength{\abovecaptionskip}{0pt}
\setlength{\belowcaptionskip}{0pt}
%\captionsetup[bi-first]{font=sf}
\newcommand{\mycaption}[3]{\bicaption[#1]{#2}{\textbf{#2}}{Table}{#3}}

\newcommand{\tablepage}[2]{\begin{minipage}{#1}\vspace{0.5ex} #2 \vspace{0.5ex}\end{minipage}}
\newcommand{\returnpage}[2]{\begin{minipage}{#1}\vspace{0.5ex} #2 \vspace{-1.5ex}\end{minipage}}

%%%%%%%%%%%%%%%%%%%%%%%%%%%%%%%%%%%%%%%%%%%%%%%%%%%%%%%%%%%%%%%%%%%%%%
%   脚注格式设置
%%%%%%%%%%%%%%%%%%%%%%%%%%%%%%%%%%%%%%%%%%%%%%%%%%%%%%%%%%%%%%%%%%%%%%
  %使用pifont包里面ding产生带圈的数字1~10
\newcommand\chnnocirc[1]{%
\ifcase#1 a \or {\ding{172}} \or {\ding{173}} \or {\ding{174}} \or {\ding{175}} \or {\ding{176}} \or {\ding{177}} \or {\ding{178}} \or {\ding{179}} \or {\ding{180}}\fi}
\renewcommand{\thefootnote}{\chnnocirc{\arabic{footnote}}}

% 自定义一个空命令,用于注释掉文本中不需要的部分。
\newcommand{\comment}[1]{}

% 双语章节重新定义BiChapter、BiSection等命令,可实现标题手动换行,但不影响目录
\def\BiChapter{\relax\@ifnextchar [{\@BiChapter}{\@@BiChapter}}
\def\@BiChapter[#1]#2#3{\chapter[#1]{#2}
    \addcontentsline{toe}{chapter}{\bfseries \xiaosi Chapter \thechapter\hspace{0.5em} #3}}
\def\@@BiChapter#1#2{\chapter{#1}
    \addcontentsline{toe}{chapter}{\bfseries \xiaosi Chapter \thechapter\hspace{0.5em}{\boldmath #2}}}

\newcommand{\BiSection}[2]
{   \section{#1}
    \addcontentsline{toe}{section}{\protect\numberline{\csname thesection\endcsname}#2}
}

\newcommand{\BiSubsection}[2]
{    \subsection{#1}
    \addcontentsline{toe}{subsection}{\protect\numberline{\csname thesubsection\endcsname}#2}
}

\newcommand{\BiSubsubsection}[2]
{    \subsubsection{#1}
    \addcontentsline{toe}{subsubsection}{\protect\numberline{\csname thesubsubsection\endcsname}#2}
}

\newcommand{\BiAppendix}[2] % 该附录命令适用于发表文章,简历等
{\phantomsection
\markboth{#1}{#1}
\addcontentsline{toc}{chapter}{\xiaosi #1}
\addcontentsline{toe}{chapter}{\bfseries \xiaosi #2}  \chapter*{#1}
}

\newcommand{\BiAppChapter}[2]    % 该附录命令适用于有章节的完整附录
{\phantomsection
 \chapter{#1}
 \addcontentsline{toe}{chapter}{\bfseries \xiaosi Appendix \thechapter~~#2}
}

\def\engcontentsname{\uppercase{CONTENTS}}
\newcommand\tableofengcontents{
   \pdfbookmark[0]{\uppercase{CONTENTS}}{encontent}
   \chapter*{\engcontentsname  %chapter*上移一行,避免在toc中出现
       \@mkboth{%
          \engcontentsname}{\engcontentsname}}
   \@starttoc{toe}%
}


%%%%%%%%%%%%%%%%%%%%%%%%%%%%%%%%%%%%%%%%%%%%%%%%%%%%%%%%%%%%%%%%%%%%%%%%%%%%%%%%
% 封面摘要
%%%%%%%%%%%%%%%%%%%%%%%%%%%%%%%%%%%%%%%%%%%%%%%%%%%%%%%%%%%%%%%%%%%%%%%%%%%%%%%%
\def\cnauthorno#1{\def\@cnauthorno{#1}}\def\@cnauthorno{}

\def\cntitle#1{\def\@cntitle{#1}}\def\@cntitle{}
\def\cnaffil#1{\def\@cnaffil{#1}}\def\@cnaffil{}
\def\cnsubject#1{\def\@cnsubject{#1}}\def\@cnsubject{}
\def\cnauthor#1{\def\@cnauthor{#1}}\def\@cnauthor{}
\def\cnsupervisor#1{\def\@cnsupervisor{#1}}\def\@cnsupervisor{}
\def\cnsupervisortitle#1{\def\@cnsupervisortitle{#1}}\def\@cnsupervisortitle{}
\def\cnsubdate#1{\def\@cnsubdate{#1}}\def\@cnsubdate{}
\def\cndefdate#1{\def\@cndefdate{#1}}\def\@cndefdate{}
\long\def\cnabstract#1{\long\def\@cnabstract{#1}}\long\def\@cnabstract{}
\def\cnkeywords#1{\def\@cnkeywords{#1}}\def\@cnkeywords{}
%\def\cnreviewer#1{\def\@cnreviewer{#1}}\def\@cnreviewer{}

\def\entitle#1{\def\@entitle{#1}}\def\@entitle{}
\def\enaffil#1{\def\@enaffil{#1}}\def\@enaffil{}
\def\ensubject#1{\def\@ensubject{#1}}\def\@ensubject{}
\def\enauthor#1{\def\@enauthor{#1}}\def\@enauthor{}
\def\ensupervisor#1{\def\@ensupervisor{#1}}\def\@ensupervisor{}
\def\ensupervisortitle#1{\def\@ensupervisortitle{#1}}\def\@ensupervisortitle{}
\def\ensubdate#1{\def\@ensubdate{#1}}\def\@ensubdate{}
\def\endefdate#1{\def\@endefdate{#1}}\def\@endefdate{}
\long\def\enabstract#1{\long\def\@enabstract{#1}}\long\def\@enabstract{}
\def\enkeywords#1{\def\@enkeywords{#1}}\def\@enkeywords{}
\def\enreviewer#1{\def\@enreviewer{#1}}\def\@enreviewer{}

\long\def\NotationList#1{\long\def\@NotationList{#1}}\long\def\@NotationList{}
\long\def\cnauthorization#1{\long\def\@cnauthorization{#1}}\long\def\@cnauthorization{}
\long\def\cninnovation#1{\long\def\@cninnovation{#1}}\long\def\@cninnovation{}

\def\natclassifiedindex#1{\def\@natclassifiedindex{#1}}\def\@natclassifiedindex{}
\def\internatclassifiedindex#1{\def\@internatclassifiedindex{#1}}\def\@internatclassifiedindex{}

\def\studentno#1{\def\@studentno{#1}}\def\@studentno{}

\def\cnstatesecrets#1{\def\@cnstatesecrets{#1}}\def\@cnstatesecrets{}
\def\enstatesecrets#1{\def\@enstatesecrets{#1}}\def\@enstatesecrets{}

\def\cnuniversty#1{\def\@cnuniversty{#1}}\def\@cnuniversty{}
\def\enuniversty#1{\def\@euniversity{#1}}\def\@euniversity{}

% 封面
\def\makecover{
  \begin{titlepage}
    %内封(扉页)
    \newpage
    \thispagestyle{empty}
    \pdfbookmark[0]{\@cntitle}{cntitlepage}
    \begin{center}
        	\renewcommand{\arraystretch}{1.5}
        	{\song \sihao
        		\begin{tabular}{@{}r@{:}l@{}}
        			分类号              & \underline{\makebox[0.35\linewidth][c]{\@natclassifiedindex}}
        	\end{tabular}}\hfill
        	{\song \sihao
        		\begin{tabular}{@{}r@{:}l@{}}
        			密 \ 级 & \underline{\makebox[0.35\linewidth][c]{}}
        	\end{tabular}}
        	
        	{\song \sihao
        		\begin{tabular}{@{}r@{:}l@{}}
        			U \hfill D \hfill C& \underline{\makebox[0.88\linewidth][c]{}} 
        	\end{tabular}}  
        	
        	\renewcommand{\arraystretch}{1}
        	
        	\vspace*{12mm}
        	
        	\centerline{\song\erhao{华~~东~~理~~工~~大~~学}}
        	\vspace*{5mm}
        	\centerline{\song\erhao{学~~位~~论~~文}}
        	%                \vspace*{2mm}
        	
        	\underline{\parbox[t][10mm][t]{.6\textwidth}{
        			\begin{center}\sihao\song{\textbf{\@cntitle}}\end{center}}}
        	
        	%	            \vspace{5mm}
        	\underline{\parbox[t][12mm][t]{.6\textwidth}{
        			\begin{center}\sihao\song{\@cnauthor}\end{center}}}
        	
        	\vspace*{4mm}
        	\hspace*{-7em} 指导教师姓名:\underline{\parbox[c][6mm][c]{.6\textwidth}{
        			\begin{center}\sihao\song{\@cnsupervisor}~~华东理工大学\end{center}}}
        	
        	\vspace*{3mm}
        	\underline{\parbox[c][2mm][c]{.6\textwidth}{
        			\begin{center}\sihao\song{}\end{center}}}            	
        	
        	\vspace*{4mm}
        	%				\hspace*{-7em}
        	申请学位级别:\underline{\makebox[7em][c]{\cnxuewei}}\hfill
        	专~~~业~~名~~~称:\underline{\makebox[7em][c]{\@cnsubject}}
        	
        	论文定稿日期:\underline{\makebox[7em][c]{2022.04.11}}\hfill
        	论文答辩日期:\underline{\makebox[7em][c]{2022.05.01}}
        	
        	\hspace*{-1.5em}学位授予单位:\underline{\makebox[22em][c]{华东理工大学}}
        	
        	\hspace*{-1.5em}学位授予日期:\underline{\makebox[22em][c]{}}
        	
        	\vspace*{4em}
        	
        	\hspace*{6em}
        	\parbox[r][30mm][c]{\textwidth}
        	{
        		\begin{center}
        			\begin{tabular}{ll}
        				答辩委员会主席:& 周扒皮~~~~~ 教~~授\\
        				评\hfill 阅\hfill 人:&周扒皮 ~~~~~教~~授\\
        				&周扒皮 ~~~~~教~~授\\
        				&周扒皮 ~~~~~教~~授\\
        				&周扒皮 ~~~~~教~~授\\
        				&周扒皮 ~~~~~教~~授\\                			
        			\end{tabular}
        		\end{center}
        	}                
      \end{center}

    % 双面打印时封面后加空白页
%    \ifoneortwosidetwoside
%      \newpage
%      ~~~\vspace{1em}
%      \thispagestyle{empty}
%    \fi

    \cleardoublepage
  \end{titlepage}
}

% 原创性与使用授权说明
\def\authorization
{
    \thispagestyle{empty}
    \@cnauthorization
    \clearpage

%    % 双面打印时加空白页
%    \ifoneortwosidetwoside
%      \newpage
%      ~~~\vspace{1em}
%      \thispagestyle{empty}
%    \fi
}

\def\innovation
{
	\thispagestyle{empty}
	\@cninnovation
	\clearpage
	
%	% 双面打印时加空白页
%	\ifoneortwosidetwoside
%	\newpage
%	~~~\vspace{1em}
%	\thispagestyle{empty}
%	\fi
}


\def\makeabstract{
%	\setcounter{page}{1}
%	\vspace*{-5em}
	\defaultfont
\begin{center}
\parbox[c][20mm][c]{\linewidth}{
	\begin{center}\sihao\hei{\@cntitle}\end{center} }  

\parbox[c][20mm][c]{\linewidth}{
	\begin{center}\sihao\song{\textbf{摘~要}}\end{center} } 
\end{center}

  {\xiaosi \song \@cnabstract}
  \vspace{5mm}

  \noindent {\hei{关键词:{\fs\@cnkeywords}}}
  \setcounter{page}{1}
  
  \defaultfont
  \cleardoublepage
  
\begin{center}
	\parbox[c][20mm][c]{\textwidth}{
		\begin{center}\sihao\hei{\textbf{\@entitle}}\end{center} }  
	
	\parbox[c][20mm][c]{\textwidth}{
		\begin{center}\sihao\song{\textbf{Abstract}}\end{center} }  
\end{center}
	\vspace{-3mm}
	
  {\xiaosi \@enabstract}
  \vspace{5mm}

  \noindent {\textbf{Key Words:}}~~{\textsf{\@enkeywords}}
  \defaultfont
  \cleardoublepage
}

\makeatletter
\def\hlinewd#1{%
  \noalign{\ifnum0=`}\fi\hrule \@height #1 \futurelet
  \reserved@a\@xhline}
\makeatother

% 定义索引生成
\def\generateindex
{
  \addcontentsline{toc}{chapter}{\indexname}
  \printindex
  \cleardoublepage
}

\raggedbottom 


%%----------  以下是论文主体,可根据内容需要进行剪裁   --------------

\begin{document}

%%----------  定义所有的图片文件在 figures 子目录下
\graphicspath{{figures/}}

%%----------  封面、摘要等文件导入
\frontmatter
\pagenumbering{Roman}             % 摘要和目录罗马字母页码

% !TEX TS-program = XeLaTeX
% !TEX encoding = UTF-8 Unicode

%%%%%%%%%%%%%%%%%%%%%%%%%%%%%%%%%%%%%%%%%%%%%%%%%%%%%%%%%%%%%%%%%%%%%%
%
%  华东理工大学本硕博论文 XeLaTeX 模版 —— 主文件 main.tex
%
%  版本:1.0.0
%  最后更新:
%  修改者:Liu Qipei, Wu Xinying, liuqipei@hrbeu.edu.cn, y10170069@mail.ecust.edu.cn
%  修订者:
%  编译环境1:macOS 10.13  + TeXLive 2020
%  编译环境2:Windows 10  + TeXLive 2020
%
%%%%%%%%%%%%%%%%%%%%%%%%%%%%%%%%%%%%%%%%%%%%%%%%%%%%%%%%%%%%%%%%%%%%%%

\natclassifiedindex{TB47}        %国内图书分类号
\internatclassifiedindex{62-5}      %国际图书分类号

\cnstatesecrets{秘密(正本)} % 保密要求 {秘密$\bigstar$10年}
\enstatesecrets{secret}

\cntitle{学位论文~XeLaTeX~模板使用说明}  % 封面用论文标题,自己可手动断行
\entitle{The Manual of XeLaTeX Thesis Template}  % 英文标题
\cnauthor{吴天天}                     % 作者姓名
\enauthor{Wu Tiantian}                  % 作者姓名 (英文)

\cnsupervisor{杨洋~~~~教授}         % 导师姓名~~~~ 职称(本科学士论文不写职称)
\cnsupervisortitle{教~~~~授}          % 导师职称
\ensupervisor{Prof. Yang yang}      % 导师职称. 姓名 (英文)
\ensupervisortitle{Professor}         %  导师职称

%\cnreviewer{周扒皮~~~~教授}            % 论文主审(评阅人)~~~~ 职称
%\enreviewer{Zhou Bapi}

\cnsubject{工业设计}                % (~按二级学科填写~)
\ensubject{Power Machine and Engineering}   % 英文二级学科名
\cnaffil{动力与能源工程学院}            % (在校生填所在系名称,同等学力人员填工作单位)
\enaffil{College of Power and Energy Engineering}

\cnuniversty{华东理工大学}             % 学校名称
\enuniversty{Harbin Engineering University}

% 论文提交于答辩日期,注意汉字和数字之间的空格。
\cnsubdate{\number\year~年~\number\month~月~\number\day~日} % 论文提交日期
\ensubdate{\today}
\cndefdate{2022~年~05~月~30~日}    % 答辩日期
\endefdate{May~1,~2022}
           % 封面信息定义
% !TEX TS-program = XeLaTeX
% !TEX encoding = UTF-8 Unicode

%%%%%%%%%%%%%%%%%%%%%%%%%%%%%%%%%%%%%%%%%%%%%%%%%%%%%%%%%%%%%%%%%%%%%%
%
%  华东理工大学本硕博论文 XeLaTeX 模版 —— 主文件 main.tex
%
%  版本:1.0.0
%  最后更新:
%  修改者:Liu Qipei, Wu Xinying, liuqipei@hrbeu.edu.cn, y10170069@mail.ecust.edu.cn
%  修订者:
%  编译环境1:macOS 10.13  + TeXLive 2020
%  编译环境2:Windows 10  + TeXLive 2020
%
%%%%%%%%%%%%%%%%%%%%%%%%%%%%%%%%%%%%%%%%%%%%%%%%%%%%%%%%%%%%%%%%%%%%%%

\cnauthorization{
    {\renewcommand{\baselinestretch}{0.2}
    \begin{center}\hei\xiaoer{学位论文使用授权声明} \end{center}
    \par}
	\renewcommand{\baselinestretch}{1.5}
    \song \xiaosan{
    \vspace{0.25cm}
    本学位论文作者完全了解学校有关保留、使用学位论文的规定,同意学校保留并向国家有关部门或机构送交论文的复印件和电子版,允许论文被查阅和借阅。本人授权华东理工大学可以将本学位论文的全部或部分内容编入有关数据库进行检索,可以采用影印、缩印或扫描等复制手段保存和汇编学位论文。保密论文在解密后遵守此规定。
    
    \noindent 论文涉密情况:
    
    \noindent$\square$ 不保密
    
    \noindent$\square$ 保密,保密期(\underline{\makebox[1.5em][c]{}}年\underline{\makebox[1.5em][c]{}}月\underline{\makebox[1.5em][c]{}}日至\underline{\makebox[1.5em][c]{}}年\underline{\makebox[1.5em][c]{}}月\underline{\makebox[1.5em][c]{}}日)
    
    \vspace*{2em}
    \noindent 学位论文作者签名: \hspace*{7.3em}  指导老师签名:
    
    \noindent 日期:\makebox[3em][c]{}年\makebox[1.5em][c]{}月\makebox[1.5em][c]{}日
     \hspace*{4em} 日期:\makebox[3em][c]{}年\makebox[1.5em][c]{}月\makebox[1.5em][c]{}日
    
    }
}
    
%    \square 去去去
    
%    \square 
    

%    \vspace{1.0cm}
%    \hspace{7.0cm}作者(签字):\hfill
%
%    \vspace{0.5cm}
%    \hspace{7.0cm}日期:\hspace{1.0cm}年\hspace{0.8cm}月\hspace{0.8cm}日
%    }


%    \vspace{1.5cm}
%    {\renewcommand\baselinestretch{0.2}
%    \begin{center}\hei\xiaoer{哈尔滨工程大学} \end{center}
%    \begin{center}\hei\xiaoer{学位论文使用授权说明}\end{center}
%    \par}
%    \song \xiaosi {
%    \vspace{0.25cm}
%    本人完全了解学校保护知识产权的有关规定,即研究生在校攻读学位期间论文工作的知识产权属于哈尔滨工程大学。哈尔滨工程大学有权保留并向国家有关部门或机构送交论文的复印件。本人允许哈尔滨工程大学将论文的部分或全部内容编入有关数据库进行检索,可采用影印、缩印或扫描等复制手段保存和汇编本学位论文,可以公布论文的全部内容。同时本人保证毕业后结合学位论文研究课题再撰写的论文一律注明作者第一署名单位为哈尔滨工程大学。涉密学位论文待解密后适用本声明。
%
%    本论文($\square$ 在授予学位后即可 \quad  $\square$ 在授予学位12个月后 \quad  $\square$ 解密后)由哈尔滨工程大学送交有关部门进行保存、汇编等。
%
%    \vspace{1.0cm}
%    \hspace{1.0cm}作者(签字):\hspace{4.5cm}导师(签字):\hfill
%
%    \vspace{0.5cm}
%    \hspace{1.0cm}日期:\hspace{1.0cm}年\hspace{0.8cm}月\hspace{0.8cm}日\hspace{2.2cm}日期:\hspace{1.0cm}年\hspace{0.8cm}月\hspace{0.8cm}日
%    }
   % 原创性声明与授权书内容
% !TEX TS-program = XeLaTeX
% !TEX encoding = UTF-8 Unicode

%%%%%%%%%%%%%%%%%%%%%%%%%%%%%%%%%%%%%%%%%%%%%%%%%%%%%%%%%%%%%%%%%%%%%%
%
%  华东理工大学本硕博论文 XeLaTeX 模版 —— 主文件 main.tex
%
%  版本:1.0.0
%  最后更新:
%  修改者:Liu Qipei, Wu Xinying, liuqipei@hrbeu.edu.cn, y10170069@mail.ecust.edu.cn
%  修订者:
%  编译环境1:macOS 10.13  + TeXLive 2020
%  编译环境2:Windows 10  + TeXLive 2020
%
%%%%%%%%%%%%%%%%%%%%%%%%%%%%%%%%%%%%%%%%%%%%%%%%%%%%%%%%%%%%%%%%%%%%%%

\cnabstract{
本模板是在参考其他高校的硕博士论文模板基础上,
并按照哈尔滨工程大学学位论文格式规范开发的~\XeLaTeX~学位论文模板,
此目前已经基本满足了论文规范的要求,而且易用性良好。不过,可能还存在着
一些问题,欢迎大家积极反馈遇到的问题,以便不断对其进行改进。

当然这个模板仅仅是一个开始,希望有更多的人能够参与进来,
不断改进准确性、易用性和较好的可维护性。

本模板的目的旨在推广~\LaTeX~这一优秀的排版软件在论文撰写中的应用,
为广大同学提供一个方便、美观的论文模板,减少论文撰写格式方面的麻烦。

本文给出了利用本模板进行论文撰写的基本步骤,并介绍了一些常用~\XeLaTeX~
排版指令。
}

\cnkeywords{
学位论文;\XeLaTeX{}模版;使用说明
}

\enabstract{
This is a \LaTeX{} template of degree thesis of Harbin Engineering University,
which is built according to the required format.
}

\enkeywords{
Degree Paper; \XeLaTeX{} Template; manual
}        % 中英文摘要内容
% !TEX TS-program = XeLaTeX
% !TEX encoding = UTF-8 Unicode

%%%%%%%%%%%%%%%%%%%%%%%%%%%%%%%%%%%%%%%%%%%%%%%%%%%%%%%%%%%%%%%%%%%%%%
%
%  华东理工大学本硕博论文 XeLaTeX 模版 —— 主文件 main.tex
%
%  版本:1.0.0
%  最后更新:
%  修改者:Liu Qipei, Wu Xinying, liuqipei@hrbeu.edu.cn, y10170069@mail.ecust.edu.cn
%  修订者:
%  编译环境1:macOS 10.13  + TeXLive 2020
%  编译环境2:Windows 10  + TeXLive 2020
%
%%%%%%%%%%%%%%%%%%%%%%%%%%%%%%%%%%%%%%%%%%%%%%%%%%%%%%%%%%%%%%%%%%%%%%

\cninnovation{
	{\renewcommand{\baselinestretch}{0.2}
		\begin{center}\hei\xiaoer{作~者~声~明}\end{center}
		\par}
	\renewcommand{\baselinestretch}{1.5}
	\song \sihao{
		\vspace{1cm}
		我郑重声明:本人恪守学术道德,崇尚严谨学风。所呈交的学位论文,是本人在导师的指导下,独立进行研究工作所取得的结果。除文中明确注明和引用的内容外,本论文不包含任何他人己经发表或撰写过的内容。论文为本人亲自撰写,并对所写内容负责。
				
		\vspace*{5em}
		\hspace*{15em} 论文作者签名:
		
		\hspace*{20em}\makebox[3em][c]{}年\makebox[1.5em][c]{}月\makebox[1.5em][c]{}日			
	}
}
        % 中英文摘要内容

\authorization  % 生成原创性与使用授权说明
\makecover      % 生成封面、扉页
\innovation    %作者声明
\makeabstract   % 生成中英文摘要


%%----------  设置目录字体和行间距
\defaultmenufont

%%----------  论文中文目录, 如不需要可用 % 注释掉下面两行
\tableofcontents
\cleardoublepage

%%----------  论文英文目录, 如不需要可用 % 注释掉下面两行
%% 如需英文目录,章节的定义需要改用\BiChapterhe和\BiSection等
%%   \BiChapter{中文章名称}{Chapter Name In English}
%%   \BiSection{中文节名称}{Section Name In Englsh}
%%   \BiSubSection{中文小节名称}{SubSection Name In Englsh}
% \tableofengcontents
% \cleardoublepage


%%%----------  插图目录, 如不需要可用 % 注释掉下面两行
%\listoffigures
%\cleardoublepage
%
%%%----------  表格目录, 如不需要可用 % 注释掉下面两行
%\listoftables
%\cleardoublepage

%%----------  设置默认正文格式
\defaultfont
\mainmatter

%%----------  正文章节, 根据需要增加

\include{body/chap01}
% !TEX TS-program = XeLaTeX
% !TEX encoding = UTF-8 Unicode

%%%%%%%%%%%%%%%%%%%%%%%%%%%%%%%%%%%%%%%%%%%%%%%%%%%%%%%%%%%%%%%%%%%%%%
%
%  华东理工大学本硕博论文 XeLaTeX 模版 —— 主文件 main.tex
%
%  版本:1.0.0
%  最后更新:
%  修改者:Liu Qipei, Wu Xinying, liuqipei@hrbeu.edu.cn, y10170069@mail.ecust.edu.cn
%  修订者:
%  编译环境1:macOS 10.13  + TeXLive 2020
%  编译环境2:Windows 10  + TeXLive 2020
%
%%%%%%%%%%%%%%%%%%%%%%%%%%%%%%%%%%%%%%%%%%%%%%%%%%%%%%%%%%%%%%%%%%%%%%

\chapter{XeLaTeX~环境配置}
\label{chap02}

\TeX{}~可以在Windows、Linux以及MacOS等操作系统下运行,鉴于大部分人都是使用Windows或Linux类操作系统,本文主要介绍着两类操作系统下的 \TeX{} 工作环境配置。

\section{Windows~操作系统}

\subsection{安装配置}
在Windows下可以使用的 \TeX{}套件有很多种,常用的有C \TeX{}和 \TeX{}Live。建议选择这两个套件中的一个使用。其中C \TeX{}只能在Windows系统下使用,而 \TeX{}Live则可以在Windows或Linux系统下使用。这两个套件都可以在网上免费下载到,建议大家下载最新的完整版本安装,因为本论文模板使用的某些宏包比较新,不然可能会造成编译错误。

\subsection{编译运行}
如果使用C \TeX{}套件的完整版,安装程序会自动配置好必须的环境变量,安装结束就可以直接使用了。

默认的,C \TeX{}安装包中会带有WinEdt软件,这是一个非常不错的 \TeX{}编辑工具。

需要注意的是,在WinEdt中必须在每个tex文件的开始添加如下的两行:
\begin{lstlisting}
  % !TEX TS-program = XeLaTeX
  % !TEX encoding = UTF-8 Unicode
\end{lstlisting}
否则文件会变成乱码。

以本模版为例,在Windows下的编译过程是这样的:
\begin{enumerate}
\item 打开main.tex文件;
\item 先点击WinEdt工具栏上的\XeLaTeX{}按钮(可能在Acrobat Reader按钮的下拉菜单
  中);
\item 再点击WinEdt工具栏上的Bib\TeX{}按钮;
\item 再点击WinEdt工具栏上的\XeLaTeX{}按钮两到三遍;
\item 最后点击WinEdt工具栏上的Acrobat Reader按钮就可以看到输出的PDF文档了。
\end{enumerate}

\section{Linux~操作系统(以~Ubuntu~为例)}
First things first,首先的工作是安装一个合适的\XeTeX{}编译系统。这个问题
并不难解决,现在主流的\LaTeX{}编译系统均已经包含了对\XeTeX{}的支持(包
括xeCJK中文宏包),并不需要自己额外再进行安装。在Linux下推荐使
用\TeX{}Live,目前最新版本为\TeX{}Live 2011。下面以在Ubuntu下的本地安装为
例,简要的说明\TeX{}Live的安装及配置过程,高玩们请主动绕行:

\begin{enumerate}
\item 下载\TeX{}live 2011镜像,点击\href{http://ftp.ctex.org/mirrors/CTAN/systems/texlive/Images/}{这里}进
  入下载列表。如果你有检查文件完整性的习惯的话,这个列表还提供了md5和sha256校验值;
\item 安装perl-tk包,以便使用图形界面进行安装。在终端中输入命
  令\texttt{\footnotesize sudo apt-get install perl-tk};
\item 挂载下载好的iso镜像,\texttt{\footnotesize sudo mkdir
    /mnt/texlive}(在~{/mnt}~下创建texlive文件夹
  ),\texttt{\footnotesize sudo mount -o loop texlive2011.iso
    /mnt/texlive}(挂载texlive2011.iso)。进入~/mnt/texlive~目录,输入命
  令~\texttt{\footnotesize sudo ./install-tl -gui}~之后出现图形界面。之后
  的操作就比较简单了,可以去掉不用的语言包以节省磁盘空间,注意选择最后一
  项Create symlinks in system directories,让安装程序自动创建语法链接。确
  定安装,耐心等待进度条到头;
\item 配置环境变量。在终端中输入~\texttt{\footnotesize sudo gedit
    /etc/bash.bashrc},在此文件末尾添加

  \begin{lstlisting}
    PATH=/usr/local/texlive/2011/bin/i386-linux: $PATH;
    export PATH
    MANPATH=/usr/local/texlive/2011/texmf/doc/man: $MANPATH;
    export MANPATH
    INFOPATH=/usr/local/texlive/2011/texmf/doc/info: $INFOPATH;
    export INFOPATH
  \end{lstlisting}

  在~{/etc/manpath.config}~文件的~\texttt{\footnotesize\# set up PATH to
    MANPATH mapping}~这行下面的列表后增加一条:
  \begin{lstlisting}
    MANPATH_MAP /usr/local/texlive/2011/bin/i386-linux
    /usr/local/texlive/2011/texmf/doc/man
  \end{lstlisting}

  在~{/etc/manpath.config}~文件的~\texttt{\footnotesize\# set up PATH to
    MANPATH mapping}~这行下面的列表后增加一条:
  \begin{lstlisting}
    MANPATH_MAP /usr/local/texlive/2011/bin/i386-linux
    /usr/local/texlive/2011/texmf/doc/man
  \end{lstlisting}
\end{enumerate}
至此安装过程结束。

以上\TeX{}Live安装过程摘自某位筒子的博客文摘,原始链接位于wordpress空间,
访问有问题,不过
\href{http://hi.baidu.com/skubuntu/blog/item/89e8de2f73a465e08a1399a3.html}{
  百度空间}有转载,虽然百度搜不着什么玩意。

接下来我们需要安装一套中文字体,你可以使用Windows下的方正、华文或者中易字
体,但要注意选择的字体最好是包含宋体、黑体、楷体和仿宋的完整套装。不过由
于这些字体在PDF浏览器中的显示效果并不好,所以建议选用Adobe的中文字体。安
装及配置过程如下:

\begin{enumerate}
\item 下载Adobe中文字体,点
  击
  \href{http://forum.ubuntu.org.cn/viewtopic.php?f=35&t=180987&start=0}{
    这里}进入下载页面;
\item 将下载的字体拷至~{/usr/share/fonts/truetype/adobe}~目录,如果没有请
  以管理员身份新建;
\item 刷新字体缓存,在终端中输入~\texttt{\footnotesize sudo fc-cache -fv }。这时,你可以通过~\texttt{\footnotesize fc-list :lang=zh-cn |sort}~命令来查看字体是否安装成功,注意fc-list后有个空格;
\item 你可能还需要在终端中运行~\texttt{\footnotesize sudo apt-get
    install poppler-data cmap-adobe-cns1 cmap-adobe-gb1}命令来解决Adobe中
  文字体在PDF文件中不显示的情况。
\end{enumerate}

这样,我们就配置好了中文字体,当然这没什么特别的,网上教程一大把。

之后我们需要一个类似于WinEdt的集成编译环境。在Ubuntu软件中心中,我们能很
容易的安装\TeX{}maker和\TeX{}works,两者功能差不多,\TeX{}maker更强大一些。
当然,你也可以自己配置VIM下的\LaTeX{}编译环境,在此就不赘述了。

\subsection{编译运行}

在安装并配置好编译环境之后,接下来的工作就是如何编译\XeLaTeX{}文件,生成
所需的PDF文档了。

任何文本编辑工具都可以用来编写论文,当然Linux下也有很多免费的集成编辑工具可以使用。

以本模版为例,在\TeX{}works编译过程是这样的:
\begin{enumerate}
\item 打开main.tex文件;
\item 将工具栏上的编译命令切换至\XeLaTeX{}后,点击运行;
\item 再将工具栏上的编译命令切换至Bib\TeX{}后,点击运行;
\item 再将工具栏上的编译命令切换至\XeLaTeX{}后,点击运行,这里需要运行两
  到三遍;
\item 如果编译没有错误,就可以看到输出的PDF文件了。
\end{enumerate}

对于\TeX{}maker,首先需要在【选项】【配置\TeX{}maker】【命令】中将第一行
的latex改成xelatex,之后用\LaTeX{}作为\XeLaTeX{}命令执行即可,其他的和上
面类似。

\section{字体}

可以使用Windows下的方正、华文或者Adobe字体\footnote{推荐使用Adobe字体,显示与打印效果比其他字体漂亮},但要注意选择的字体最好是包含宋体、黑体、楷体和仿宋的完整套装。不过由于这些字体在PDF浏览器中的显示效果并不好,所以建议选用Adobe的中文字体。

本模板默认是使用Adobe库,因此在使用此模板撰写论文前,应该安装相应的字库。在Windows操作系统下,只要把字库文件复制的Windows \textbackslash Fonts文件夹下即可,而对于Linux系统,可通过右键点击字库文件然后选择【安装字库】菜单选项进行安装。Linux对于系统新安装的字库,需要使用命令~sudo fc-cache -fsv:刷新缓存后才可以使用。

本模板使用的字库有:

\begin{enumerate}
\item Adobe 楷体
\item Adobe 黑体
\item Adobe 宋体
\item Adobe 仿宋
\end{enumerate}

系统中应该安装的英文字体:
\begin{enumerate}
\item Times New Roman
\item Consolas
\end{enumerate}


\section*{本章小结}
\LaTeX{}~工作环境安装与配置简介。
% !TEX TS-program = XeLaTeX
% !TEX encoding = UTF-8 Unicode

%%%%%%%%%%%%%%%%%%%%%%%%%%%%%%%%%%%%%%%%%%%%%%%%%%%%%%%%%%%%%%%%%%%%%%
%
%  华东理工大学本硕博论文 XeLaTeX 模版 —— 主文件 main.tex
%
%  版本:1.0.0
%  最后更新:
%  修改者:Liu Qipei, Wu Xinying, liuqipei@hrbeu.edu.cn, y10170069@mail.ecust.edu.cn
%  修订者:
%  编译环境1:macOS 10.13  + TeXLive 2020
%  编译环境2:Windows 10  + TeXLive 2020
%
%%%%%%%%%%%%%%%%%%%%%%%%%%%%%%%%%%%%%%%%%%%%%%%%%%%%%%%%%%%%%%%%%%%%%%

\chapter{模版使用说明}
\label{chap03}

\section{个人信息}
使用模版的第一步当然是修改您的个人信息。与个人信息有关的内容位
于~{/preface/cover.tex}~文件中。对照着模版内容改就好了,没有什么难度。填
写专业、姓名和导师的时候注意添加适当空格,也就是$\sim$字符,以保持段落对齐。
这里默认论文提交日期为最后一次编译~main.tex~的日期,答辩日期需要手工设置。

\section{模版设置}

模板设置包括选择论文的学位类型、学科类型、汉字库和打印方式等,
这些内容的设置在~main.tex~文件中通过修改~\verb|\def|~命令实现。
\begin{itemize}
\item 学位论文类型选择 \\
学位论文类型可以是:~Doctor~(博士)、~Master~(硕士)和~Bachelor(学士)。
如论文选择“硕士”论文模板,则学位论文类型选择定义为:
\verb|\def\xuewei{Master}|

\item 定义学科 \\
本模板定义的学科包括:
\begin{table}[htbp]
  \bicaption[tab:xueke]{学科定义1}{学科定义}{Tab.}{Subject Definition}
  \centering
  \vspace{0.2cm}
  \zhongwu
  \begin{tabular}{cc}
    \toprule
    学科定义  & 学科类型  \\
    \midrule
    Engineering   & 工学 \\
    Science       & 科学 \\
    Management    & 管理 \\
    Arts          & 艺术 \\
    Philosophy    & 哲学 \\
    Economics     & 经济 \\
    Laws          & 法律 \\
    Education     & 教育 \\
    History       & 历史 \\
    \bottomrule
  \end{tabular}
\end{table}

如选择工学学科,则学科类型定义为:
\verb|\def\xueke{Engineering}|

\item 选择字体 \\
选择字体库,包括:\\
adobefont --Adobe 汉字库 \\
windowsfont --Windows 系统汉字库 \\
linuxfont --Linux 系统字库 \\

如选择~Adobe~汉字库,则选择类型选择定义为:
\verb|\def\fontselect{adobefont}|

\item 打印方式选择
论文打印方式包括~oneside~(单面打印)和~twoside~(双面打印)。
如选择双面打印方式,则打印方式选择定义为:
\verb|\def\oneortwoside{twoside}|
\end{itemize}

\section{中英文摘要、关键字}
中英文摘要和关键字也位于~{/preface/cover.tex}~文件中,分别定义
在cnabstract、 enabstract、cnkeywords和enkeywords中,替换成自己的即可。

这里附上研究生院对摘要和关键字的要求:
\begin{asparaenum}
\item “摘要”是摘要部分的标题,不可省略。论文摘要是学位论文的缩影,文字
  要简练、明确。内容要包括目的、方法、结果和结论。单位制一律换算成国际标
  准计量单位制,除特殊情况外,数字一律用阿拉伯数码。文中不允许出现插图,
  重要的表格可以写入;
\item 关键词请尽量用《汉语主题词表》等词表提供的规范词。关键词之间用全角
  分号间隔,末尾不加标点;
\item 英文摘要和中文摘要对应,但不要逐字翻译。英文关键字使用半角分号间隔,
  末尾同样不加标点。
\end{asparaenum}

\section{正文}

正文部分包括了绪论(chap01.tex)、正文内容章节
(chap02.tex、chap03.tex、chap04.tex、……)、结论(conclusion.tex)三个部分,
均位于body文件夹中。同时位于body文件夹下的还有Bib\TeX{}参考文献文件
(reference.bib)。

正文内容章节以chapXX.tex形式为文件名,从01开始计数,使得文件名序号即为章
节序号。这些正文内容章节需要依次写入main.tex文件中,格式
为~\verb|\include{body/chapXX}| 。

所有的图片放在figure文件夹中。

下面是研究生院对正文的要求:

“正文”不可省略。

正文是硕士学位论文的主体,要着重反映研究生自己的工作,要突出新的见解,例
如新思想、新观点、新规律、新研究方法、新结果等。正文一般可包括:理论分析;
试验装置和测试方法;对试验结果的分析讨论及理论计算结果的比较等。

正文要求论点正确,推理严谨,数据可靠,文字精练,条理分明,文字图表清晰整
齐,计算单位采用国务院颁布的《统一公制计量单位中文名称方案》中规定和名称。
各类单位、符号必须在论文中统一使用,外文字母必须注意大小写,正斜体。简化
字采用正式公布过的,不能自造和误写。利用别人研究成果必须附加说明。引用前
人材料必须引证原著文字。在论文的行文上,要注意语句通顺,达到科技论文所必
须具备的“正确、准确、明确”的要求。

\section{格式设置}
一般来说,采用本模板后不需要另外使用字体、字号、颜色等文字格式设置操作,
模板会根据内容自动选用合适的格式。但在某些情况下,如果需要特殊设置字体、
字号与颜色,那么可以使用下面这些方法进行设置。

\subsection{字体设置}
本模板预定义的汉字字体包括:{\song 宋体}、{\hei 黑体}、{\kai 楷体}和{\fs 仿宋},
每种字体还包括正体、斜体、粗体,而且可以实现复合效果,例如:\\
{\song 宋体  \textbf{加粗宋体} \textsl{斜体宋体} \textbf{\textsl{加粗斜体宋体}}} \\
{\hei 黑体   \textbf{加粗黑体} \textsl{斜体黑体} \textbf{\textsl{加粗斜体黑体}}} \\
{\kai 楷体   \textbf{加粗楷体} \textsl{斜体楷体} \textbf{\textsl{加粗斜体楷体}}} \\
{\fs  仿宋   \textbf{加粗仿宋} \textsl{斜体仿宋} \textbf{\textsl{加粗斜体仿宋}}} \\

设置字体的方法是在需要修改字体的文字前面加入字体定义指令,格式为\textbackslash~font,
其中\textbackslash~song表示宋体,\textbackslash~hei表示黑体,\textbackslash~kai表示楷体,
\textbackslash~fs表示仿宋,\textbackslash~xkai表示行楷。粗体的格式化指令为\textbackslash~textbf,
斜体的格式化指令为\textbackslash~textsl。
另外,可以用\{~~\}限定字体的设置范围,及将字体格式化指令和文字内容都放到\{~~\}内,
这样括号外面的内容格式自动恢复为以前的格式。

上面字体显示效果的实现代码为:

\begin{lstlisting}
{\song 宋体  \textbf{加粗宋体} \textsl{斜体宋体}
 \textbf{\textsl{加粗斜体宋体}}} \\
{\hei 黑体   \textbf{加粗黑体} \textsl{斜体黑体}
 \textbf{\textsl{加粗斜体黑体}}} \\
{\kai 楷体   \textbf{加粗楷体} \textsl{斜体楷体}
 \textbf{\textsl{加粗斜体楷体}}} \\
{\fs  仿宋   \textbf{加粗仿宋} \textsl{斜体仿宋}
 \textbf{\textsl{加粗斜体仿宋}}} \\
\end{lstlisting}

\subsection{字号设置}
设置字号的方法为:
\begin{lstlisting}
\xiaowu   \textbackslash xiaowu~  小五,默认单倍行距 \\
\wuhao    \textbackslash wuhao~   五号,默认单倍行距 \\
\zhongwu  \textbackslash zhongwu~ 五号,默认1.5 倍行距 \\
\dawu     \textbackslash dawu~    五号,默认1.75倍行距 \\
\xiaosi   \textbackslash xiaosi~  小四,默认1.25倍行距 \\
\daxiaosi \textbackslash daxiaosi~小四,默认1.5 倍行距 \\
\sihao    \textbackslash sihao~   四号,默认1.25倍行距 \\
\xiaosan  \textbackslash xiaosan~ 小三,默认1.25 倍行距 \\
\sanhao   \textbackslash sanhao~  三号,默认1.25 倍行距 \\
\xiaoer   \textbackslash xiaoer~  小二,默认1.25 倍行距 \\
\erhao    \textbackslash erhao~   二号,默认1.25 倍行距 \\
\xiaoyi   \textbackslash xiaoyi~  小一,默认1.25 倍行距 \\
\yihao    \textbackslash yihao~   一号,默认1.5  倍行距 \\
\end{lstlisting}

打印效果如下:

 \begin{flushleft}
 {
\xiaowu   \textbackslash xiaowu~  小五,默认单倍行距 \\
\wuhao    \textbackslash wuhao~   五号,默认单倍行距 \\
\zhongwu  \textbackslash zhongwu~ 五号,默认1.5 倍行距 \\
\dawu     \textbackslash dawu~    五号,默认1.75倍行距 \\
\xiaosi   \textbackslash xiaosi~  小四,默认1.25倍行距 \\
\daxiaosi \textbackslash daxiaosi~小四,默认1.5 倍行距 \\
\sihao    \textbackslash sihao~   四号,默认1.25倍行距 \\
\xiaosan  \textbackslash xiaosan~ 小三,默认1.25 倍行距 \\
\sanhao   \textbackslash sanhao~  三号,默认1.25 倍行距 \\
\xiaoer   \textbackslash xiaoer~  小二,默认1.25 倍行距 \\
\erhao    \textbackslash erhao~   二号,默认1.25 倍行距 \\
\xiaoyi   \textbackslash xiaoyi~  小一,默认1.25 倍行距 \\
\yihao    \textbackslash yihao~   一号,默认1.5  倍行距 \\
}
\end{flushleft}

\subsection{颜色设置}
设置文字颜色的方法为:
\begin{lstlisting}
\definecolor {myrgb}{rgb}{0.25, 0.5, 0.25}
\definecolor {mycmyk}{cmyk}{1, 0.8, 0.2, 0.1}

\hei \textcolor{black}  {这是预定义颜色-黑色 balck}  \\
\hei \textcolor{red}    {这是预定义颜色-红色 red}  \\
\hei \textcolor{blue}   {这是预定义颜色-蓝色 blue}  \\
\hei \textcolor{yellow} {这是预定义颜色-黄色 yellow}  \\
\hei \textcolor{myrgb}  {这是自定义RGB颜色 myrgb}  \\
\hei \textcolor{mycmyk} {这是自定义CMYK颜色 mycmyk}  \\
\end{lstlisting}


文字颜色设置打印效果:
\begin{flushleft}
\xiaosan
{
    \definecolor {myrgb}{rgb}{0.25, 0.5, 0.25}
    \definecolor {mycmyk}{cmyk}{1, 0.8, 0.2, 0.1}

    \hei \textcolor{black}  {这是预定义颜色-黑色 balck}  \\
    \hei \textcolor{red}    {这是预定义颜色-红色 red}  \\
    \hei \textcolor{blue}   {这是预定义颜色-蓝色 blue}  \\
    \hei \textcolor{yellow} {这是预定义颜色-黄色 yellow}  \\
    \hei \textcolor{myrgb}  {这是自定义RGB颜色 myrgb}  \\
    \hei \textcolor{mycmyk} {这是自定义CMYK颜色 mycmyk}  \\
}
\end{flushleft}

\section*{本章小结}
简单介绍模板使用方法和文字格式化方法。

\include{body/chap04}
\include{body/chap05}
\include{body/chap06}
% !TEX TS-program = XeLaTeX
% !TEX encoding = UTF-8 Unicode

%%%%%%%%%%%%%%%%%%%%%%%%%%%%%%%%%%%%%%%%%%%%%%%%%%%%%%%%%%%%%%%%%%%%%%
%
%  华东理工大学本硕博论文 XeLaTeX 模版 —— 主文件 main.tex
%
%  版本:1.0.0
%  最后更新:
%  修改者:Liu Qipei, Wu Xinying, liuqipei@hrbeu.edu.cn, y10170069@mail.ecust.edu.cn
%  修订者:
%  编译环境1:macOS 10.13  + TeXLive 2020
%  编译环境2:Windows 10  + TeXLive 2020
%
%%%%%%%%%%%%%%%%%%%%%%%%%%%%%%%%%%%%%%%%%%%%%%%%%%%%%%%%%%%%%%%%%%%%%%

\chapter{参考文献}
\label{chap07}

参考文献的引用一般有两种方式,即行间引用和上标引用。

行间引用使用\textbackslash lcite~\{~~\}~语句实现,其显示效果是这样的:例如文献\lcite{DXM2005}论述了什么什么,而文献\lcite{OJP1999,kelton2002,strawderman2001,LQL1999}则对这个那个进行了研究。

上标引用使用\textbackslash cite~\{~~\}~语句实现,下面这段文字是普通的上标引用格式

我们的一切知识都是从经验开始\cite{LQL1999},这是没有任何怀疑的\cite{DXM2005}\cite{DXM2000};
因为,如果不是对象激动我们的感官,一则由它们自己引起表象,一则使我们的知性活动运作起来,对这些表象加
以比较,把它们粘结或分开,\cite{OJP1999,OJP1991}这样把感性印象的原始素材加工成称之为经验的对象
知识,那么知识能力又该由什么来唤起活动呢?\cite{braun2007,kelton2002,strawderman2001,LQL1999}所以
按照时间,我们没有任何知识是先行于经验的,一切知识都是从经验开始的。

只要是中文文献,图书,期刊,会议,专利等等需要为每个条目增加一个域:
\begin{lstlisting}
  language={CN},
\end{lstlisting}

对于\cite{DXM2005}参考文献\cite{OJP1999},原先的bib文件是\scite{OJP1991}这样的:
\begin{lstlisting}
  @article{ LQL1999 ,
    title={ 康德何以步安瑟尔谟的后尘? },
    author={ 李秋零 },
    journal={ 中国人民大学学报 },
    volume={2},
    year={1999}
  }
\end{lstlisting}


但是由于是中文文献,需要增加一个语言域,就变成下列样式:
\begin{lstlisting}
  @article{ LQL1999,
    title={ 康德何以步安瑟尔谟的后尘? },
    author={ 李秋零 },
    language={CN},
    journal={ 中国人民大学学报 },
    volume={2},
    year={1999}
  }
\end{lstlisting}

\section{~BibTeX~文献文件的写法}

用在~\LaTeX~中的~\textsc{Bib}\kern-.08em\TeX~文献文件的扩展名为~bib,此模板中,该文件即为~reference.bib。bibtex.exe 命令根据~GBT7714-2005NLang-HIT.bst 文件定义的文献格式,将~reference.bib 中的文献数据转换为输出文档中的文献列表。GBT7714-2005NLang-HIT.bst 文件是在~\href{http://bbs.ctex.org/attachment.php?aid=MjA3MDh8ZDcyMjc2MTN8MTMyNTYzNjY4OHxhZTg4bkNCUVJiRzA0WmU3TmlMbVdTUVExa0xtV2puWWc0dkdqbVJhbTVMdy9mVQ\%3D\%3D}{GBT7714-2005NLang-UTF8.bst} 文件的基础上修改得到的,所做的唯一一处改动是将姓氏字母全部大写的英文作者名改为只首字母大写,以保证和\href{http://219.217.226.141/xuewei/guifan.doc}{《研究生学位论文撰写规范》}及其\href{http://219.217.226.141/xuewei/fanli.doc}{《研究生学位论文书写范例》}相一致。

bib 文件的编写方法可参考模板中已给出的例子,也可参考~\href{http://bbs.ctex.org/attachment.php?aid=MTk3OTd8NjY1ODc5OGV8MTMyNTY0MTEyMnxhZGZkYWpsa0I2RGZwNDR5Z1lyeStjb1dKRS8rTnJub3lvT2FkNDNJbHl1UWVkVQ\%3D\%3D}{GBT7714-2005.bst 说明文档20060919
} 中所给出的例子。

中文文献需要添加一个额外的~language 域,并使得域值非空,这样~bst 文件就能够判断此文献为中文文献,进而能正确地生成参考文献格式。

GBT7714-2005.bst 对于国标~GB/T 7714-2005 的文献分类如表~\ref{tab:entrytypes} 所示。对于每种文献类型的缺省类型,已经设置好相应的文献标识码,因此不需要输入相应的文献
标识码。扩展类型的文献则应再添加一个~TypeofLit 域,并需要将其域值改为相应的文献标识码。
\begin{table}[htbp]
\bicaption[tab:entrytypes]{}{GBT7714-2005.bst 的分类方式}{Table$\!$}{Classification method of GBT7714-2005.bst}
\vspace{0.5em}\centering\wuhao
\begin{tabular}{llll}
\toprule[1.5pt]
文献类型 & 缺省类型 & 扩展类型(需要手 & 主要特征\\
 &  & 工加入文献标识码) & \\
\midrule[1pt]
article & 文章[J] & 报纸中的析出文献[N] & 年,卷(期):页码\\
 &  & 在线文章[J/OL] & \\
book & 书[M] & 论文集、会议录[C] & \\
 &  & 在线书[M/OL] & \\
 &  & 汇编[G] & \\
inbook & 书的某几页[M] &  & \\
incollection & 书中析出的文章[M]// & 汇编的析出文献[G]// & 析出文献[文献标识码]//\\
 &  & 标准的析出文献[S]// & \\
proceedings &  &  & \\
inproceedings & 论文集、会议录中的 & 在线论文集、 & 析出文献[文献标识码]//\\
/conference & 析出文献[C]// & 会议录[C/OL]// & \\
mastersthesis & 毕业论文[D] &  & 类似book类\\
phdthesis & 毕业论文[D] &  & 类似book类\\
techreport & 科技报告[R] &  & 类似book类\\
misc &  & 杂项[],例如:专利[P] & 此类一般是网上文件,\\
 &  & 网上专利[P/OL] & 按照国标规定顺序\\
 &  & 网上电子公告[EB/OL] & 编码制时不输出年份\\
 &  & 磁盘[CP/DK] & \\
\bottomrule[1.5pt]
\end{tabular}
\end{table}

《研究生学位论文撰写规范》及《研究生学位论文书写范例》中所列英文参考文献例子中的文章名的每个实词首字母都大写,因此需要将英文参考文献的~title 域手动修改为每个实词首字母大写。

英文参考文献在~author 域中的作者名需要将姓置前,名置后。

\section{参考文献的引用}

需要将~main.tex 文件中的语句~\verb|\nocite{*}| 屏蔽掉,这样,文中未引用的参考文献就不会出现在文后的参考文献列表中。文中参考文献的引用方法:

\begin{itemize}
\item 行文引用请使用命令~\verb|\lcite{引用词}|,引用效果为“\lcite{OJP1999}”;
\item 上标引用请使用命令~\verb|\cite{引用词}|,引用效果为“\cite{OJP1999}”。
\end{itemize}
其中,上标引用命令~\verb|\lcite{}| 为本模板自定义的命令,其定义为
\begin{verbatim}
\DeclareRobustCommand\lcite{\@lcite}
\def\@lcite#1{\begingroup\let\@cite\NAT@citenum\citep{#1}\endgroup}
\end{verbatim}

\section*{本章小结}
参考文献排版方法介绍。


\cleardoublepage
%%----------  结论, 一般都需要 , 如不需要可用 % 注释掉下面两行
% !TEX TS-program = XeLaTeX
% !TEX encoding = UTF-8 Unicode

%%%%%%%%%%%%%%%%%%%%%%%%%%%%%%%%%%%%%%%%%%%%%%%%%%%%%%%%%%%%%%%%%%%%%%
%
%  华东理工大学本硕博论文 XeLaTeX 模版 —— 主文件 main.tex
%
%  版本:1.0.0
%  最后更新:
%  修改者:Liu Qipei, Wu Xinying, liuqipei@hrbeu.edu.cn, y10170069@mail.ecust.edu.cn
%  修订者:
%  编译环境1:macOS 10.13  + TeXLive 2020
%  编译环境2:Windows 10  + TeXLive 2020
%
%%%%%%%%%%%%%%%%%%%%%%%%%%%%%%%%%%%%%%%%%%%%%%%%%%%%%%%%%%%%%%%%%%%%%%

\appendix{结  论}

结论是理论分析和实验结果的逻辑发展,是整篇论文的归宿。
结论是在理论分析、试验结果的基础上,经过分析、推理、判断、归纳的过程而形成的总观点。
结论必须完整、准确、鲜明、并突出与前人不同的新见解。
\cleardoublepage

%%----------  正文后附加内容, 包括参考文献、致谢等等, 根据需要选择
\backmatter

%%----------  参考文献, 文献数据库为 reference/reference.bib
\wuhao  % 设置参考文献字号为五号
\bibliographystyle{GBT7714-2005}
\bibliography{reference/reference}
\addcontentsline{toc}{chapter}{参考文献}
\cleardoublepage
\defaultfont

%%----------  发表的文章列表, 如不需要可用 % 注释掉下面两行
%% !TEX TS-program = XeLaTeX
% !TEX encoding = UTF-8 Unicode

%%%%%%%%%%%%%%%%%%%%%%%%%%%%%%%%%%%%%%%%%%%%%%%%%%%%%%%%%%%%%%%%%%%%%%
%
%  华东理工大学本硕博论文 XeLaTeX 模版 —— 主文件 main.tex
%
%  版本:1.0.0
%  最后更新:
%  修改者:Liu Qipei, Wu Xinying, liuqipei@hrbeu.edu.cn, y10170069@mail.ecust.edu.cn
%  修订者:
%  编译环境2:Windows 10  + TeXLive 2021/2020
%
%%%%%%%%%%%%%%%%%%%%%%%%%%%%%%%%%%%%%%%%%%%%%%%%%%%%%%%%%%%%%%%%%%%%%%

\appendix{攻读学位期间发表学术论文情况}

仅列出博士生攻读学位期间发表与学位论文有关的学术论文,
并注明属于学位论文内容的部分(章节),
所有作者及其顺序、所发表的刊物名称(包括主办单位、是否被SCI、EI检索期刊)、时间、期号与页码。
其他时间或与学位论文内容(章节)无关的论文不得列出。示例如下:

% \renewcommand{\labelenumi}{[\arabic{enumi}]}
\section*{在国际和国内学术刊物上发表的论文}
\begin{enumerate}[label={[\arabic*]}]
\item L. Wang, S. Kang, H. Shum, G. Xu, Error Analysis of Pure
  Rotation-based Self-Calibration, {\em{IEEE Transactions on Pattern
      Analysis and Machine Intelligence (PAMI)}}, in press
\item ×××,××,×××. 一种基于全景图的三维房间导航方法.
  软件学报, 2002, 13(Suppl.): 31-35
\end{enumerate}

\section*{在国际和国内学术会议上发表的论文}
\begin{enumerate}[label={[\arabic*]}]
\item L. Wang, S. Kang, H. Shum, G. Xu, Error Analysis of Pure
  Rotation-based Self-Calibration, {\em{in Proceedings of the Eighth
      IEEE International Conference on Computer Vision(ICCV'01)}}, I:
  464-471, Vancouver, BC, Canada, July, 2001
\item L. Wang, X. Liu, L. Xia, G. Xu, A. Bruckstein, Image
  Orientation Detection with Integrated Human Perception Cues,
  {\em{in Proceedings of IEEE International Conference on Image
      Processing (ICIP'03)}}, in press
\end{enumerate}


%\cleardoublepage

%%----------  致谢, 一般都需要, 如不需要可用 % 注释掉下面两行
% !TEX TS-program = XeLaTeX
% !TEX encoding = UTF-8 Unicode

%%%%%%%%%%%%%%%%%%%%%%%%%%%%%%%%%%%%%%%%%%%%%%%%%%%%%%%%%%%%%%%%%%%%%%
%
%  华东理工大学本硕博论文 XeLaTeX 模版 —— 主文件 main.tex
%
%  版本:1.0.0
%  最后更新:
%  修改者:Liu Qipei, Wu Xinying, liuqipei@hrbeu.edu.cn, y10170069@mail.ecust.edu.cn
%  修订者:
%  编译环境1:macOS 10.13  + TeXLive 2020
%  编译环境2:Windows 10  + TeXLive 2020
%
%%%%%%%%%%%%%%%%%%%%%%%%%%%%%%%%%%%%%%%%%%%%%%%%%%%%%%%%%%%%%%%%%%%%%%

\appendix{致  谢}

学位论文中不得书写与论文工作无关的人和事,对导师的致谢要实事求是。

一同工作的同志对本研究所做的贡献应在论文中做明确的说明并表示谢意。

这部分内容不可省略。

在这里,向所有协助测试的同学、朋友表示感谢。

\cleardoublepage

%%----------  作者简介, 如不需要可用 % 注释掉下面两行
% !TEX TS-program = XeLaTeX
% !TEX encoding = UTF-8 Unicode

%%%%%%%%%%%%%%%%%%%%%%%%%%%%%%%%%%%%%%%%%%%%%%%%%%%%%%%%%%%%%%%%%%%%%%
%
%  华东理工大学本硕博论文 XeLaTeX 模版 —— 主文件 main.tex
%
%  版本:1.0.0
%  最后更新:
%  修改者:Liu Qipei, Wu Xinying, liuqipei@hrbeu.edu.cn, y10170069@mail.ecust.edu.cn
%  修订者:
%  编译环境2:Windows 10  + TeXLive 2021/2020
%
%%%%%%%%%%%%%%%%%%%%%%%%%%%%%%%%%%%%%%%%%%%%%%%%%%%%%%%%%%%%%%%%%%%%%%

\appendix{作者简介}

\begin{window}[0,r,{\mbox{\includegraphics[width=3.5cm]{author.jpg}}},{}]
\end{window}

姓名:吴 天天

性别:女

出生年月:1996~年~06~月~01~日

民族:汉

籍贯:上海市

研究方向:工业设计

座右铭:世上无难事,只要肯放弃

\vspace*{5em}

简历:

从这里开始写简历

200X.9-200X.7  XX大学XX专业个人简历,从大学起。

200X.9-200X.7  XX大学XX专业个人简历,从大学起。

200X.9-200X.7  XX大学XX专业个人简历,从大学起。

200X.9-200X.7  XX大学XX专业个人简历,从大学起。

200X.9-200X.7  XX大学XX专业个人简历,从大学起。

\cleardoublepage

%%----------  其他附录, 如不需要可用 % 注释掉下面几行
% !TEX TS-program = XeLaTeX
% !TEX encoding = UTF-8 Unicode

%%%%%%%%%%%%%%%%%%%%%%%%%%%%%%%%%%%%%%%%%%%%%%%%%%%%%%%%%%%%%%%%%%%%%%
%
%  华东理工大学本硕博论文 XeLaTeX 模版 —— 主文件 main.tex
%
%  版本:1.0.0
%  最后更新:
%  修改者:Liu Qipei, Wu Xinying, liuqipei@hrbeu.edu.cn, y10170069@mail.ecust.edu.cn
%  修订者:
%  编译环境1:macOS 10.13  + TeXLive 2020
%  编译环境2:Windows 10  + TeXLive 2020
%
%%%%%%%%%%%%%%%%%%%%%%%%%%%%%%%%%%%%%%%%%%%%%%%%%%%%%%%%%%%%%%%%%%%%%%

\appendix{附录~A~~附录内容名称}

以下内容可放在附录之内:

\begin{enumerate}
\item 正文内过于冗长的公式推导;
\item 方便他人阅读所需的辅助性数学工具或表格;
\item 重复性数据和图表;
\item 论文使用的主要符号的意义和单位;
\item 程序说明和程序全文。
\end{enumerate}

这部分内容可省略。


%\cleardoublepage


%%%%%%%%%%%% 到这就结束了! %%%%%%%%%%%%%%%%%%%%%%%%%%%%
\end{document}
